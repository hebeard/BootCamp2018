
% Default to the notebook output style

    


% Inherit from the specified cell style.




    
\documentclass[11pt]{article}

    
    
    \usepackage[T1]{fontenc}
    % Nicer default font (+ math font) than Computer Modern for most use cases
    \usepackage{mathpazo}

    % Basic figure setup, for now with no caption control since it's done
    % automatically by Pandoc (which extracts ![](path) syntax from Markdown).
    \usepackage{graphicx}
    % We will generate all images so they have a width \maxwidth. This means
    % that they will get their normal width if they fit onto the page, but
    % are scaled down if they would overflow the margins.
    \makeatletter
    \def\maxwidth{\ifdim\Gin@nat@width>\linewidth\linewidth
    \else\Gin@nat@width\fi}
    \makeatother
    \let\Oldincludegraphics\includegraphics
    % Set max figure width to be 80% of text width, for now hardcoded.
    \renewcommand{\includegraphics}[1]{\Oldincludegraphics[width=.8\maxwidth]{#1}}
    % Ensure that by default, figures have no caption (until we provide a
    % proper Figure object with a Caption API and a way to capture that
    % in the conversion process - todo).
    \usepackage{caption}
    \DeclareCaptionLabelFormat{nolabel}{}
    \captionsetup{labelformat=nolabel}

    \usepackage{adjustbox} % Used to constrain images to a maximum size 
    \usepackage{xcolor} % Allow colors to be defined
    \usepackage{enumerate} % Needed for markdown enumerations to work
    \usepackage{geometry} % Used to adjust the document margins
    \usepackage{amsmath} % Equations
    \usepackage{amssymb} % Equations
    \usepackage{textcomp} % defines textquotesingle
    % Hack from http://tex.stackexchange.com/a/47451/13684:
    \AtBeginDocument{%
        \def\PYZsq{\textquotesingle}% Upright quotes in Pygmentized code
    }
    \usepackage{upquote} % Upright quotes for verbatim code
    \usepackage{eurosym} % defines \euro
    \usepackage[mathletters]{ucs} % Extended unicode (utf-8) support
    \usepackage[utf8x]{inputenc} % Allow utf-8 characters in the tex document
    \usepackage{fancyvrb} % verbatim replacement that allows latex
    \usepackage{grffile} % extends the file name processing of package graphics 
                         % to support a larger range 
    % The hyperref package gives us a pdf with properly built
    % internal navigation ('pdf bookmarks' for the table of contents,
    % internal cross-reference links, web links for URLs, etc.)
    \usepackage{hyperref}
    \usepackage{longtable} % longtable support required by pandoc >1.10
    \usepackage{booktabs}  % table support for pandoc > 1.12.2
    \usepackage[inline]{enumitem} % IRkernel/repr support (it uses the enumerate* environment)
    \usepackage[normalem]{ulem} % ulem is needed to support strikethroughs (\sout)
                                % normalem makes italics be italics, not underlines
    

    
    
    % Colors for the hyperref package
    \definecolor{urlcolor}{rgb}{0,.145,.698}
    \definecolor{linkcolor}{rgb}{.71,0.21,0.01}
    \definecolor{citecolor}{rgb}{.12,.54,.11}

    % ANSI colors
    \definecolor{ansi-black}{HTML}{3E424D}
    \definecolor{ansi-black-intense}{HTML}{282C36}
    \definecolor{ansi-red}{HTML}{E75C58}
    \definecolor{ansi-red-intense}{HTML}{B22B31}
    \definecolor{ansi-green}{HTML}{00A250}
    \definecolor{ansi-green-intense}{HTML}{007427}
    \definecolor{ansi-yellow}{HTML}{DDB62B}
    \definecolor{ansi-yellow-intense}{HTML}{B27D12}
    \definecolor{ansi-blue}{HTML}{208FFB}
    \definecolor{ansi-blue-intense}{HTML}{0065CA}
    \definecolor{ansi-magenta}{HTML}{D160C4}
    \definecolor{ansi-magenta-intense}{HTML}{A03196}
    \definecolor{ansi-cyan}{HTML}{60C6C8}
    \definecolor{ansi-cyan-intense}{HTML}{258F8F}
    \definecolor{ansi-white}{HTML}{C5C1B4}
    \definecolor{ansi-white-intense}{HTML}{A1A6B2}

    % commands and environments needed by pandoc snippets
    % extracted from the output of `pandoc -s`
    \providecommand{\tightlist}{%
      \setlength{\itemsep}{0pt}\setlength{\parskip}{0pt}}
    \DefineVerbatimEnvironment{Highlighting}{Verbatim}{commandchars=\\\{\}}
    % Add ',fontsize=\small' for more characters per line
    \newenvironment{Shaded}{}{}
    \newcommand{\KeywordTok}[1]{\textcolor[rgb]{0.00,0.44,0.13}{\textbf{{#1}}}}
    \newcommand{\DataTypeTok}[1]{\textcolor[rgb]{0.56,0.13,0.00}{{#1}}}
    \newcommand{\DecValTok}[1]{\textcolor[rgb]{0.25,0.63,0.44}{{#1}}}
    \newcommand{\BaseNTok}[1]{\textcolor[rgb]{0.25,0.63,0.44}{{#1}}}
    \newcommand{\FloatTok}[1]{\textcolor[rgb]{0.25,0.63,0.44}{{#1}}}
    \newcommand{\CharTok}[1]{\textcolor[rgb]{0.25,0.44,0.63}{{#1}}}
    \newcommand{\StringTok}[1]{\textcolor[rgb]{0.25,0.44,0.63}{{#1}}}
    \newcommand{\CommentTok}[1]{\textcolor[rgb]{0.38,0.63,0.69}{\textit{{#1}}}}
    \newcommand{\OtherTok}[1]{\textcolor[rgb]{0.00,0.44,0.13}{{#1}}}
    \newcommand{\AlertTok}[1]{\textcolor[rgb]{1.00,0.00,0.00}{\textbf{{#1}}}}
    \newcommand{\FunctionTok}[1]{\textcolor[rgb]{0.02,0.16,0.49}{{#1}}}
    \newcommand{\RegionMarkerTok}[1]{{#1}}
    \newcommand{\ErrorTok}[1]{\textcolor[rgb]{1.00,0.00,0.00}{\textbf{{#1}}}}
    \newcommand{\NormalTok}[1]{{#1}}
    
    % Additional commands for more recent versions of Pandoc
    \newcommand{\ConstantTok}[1]{\textcolor[rgb]{0.53,0.00,0.00}{{#1}}}
    \newcommand{\SpecialCharTok}[1]{\textcolor[rgb]{0.25,0.44,0.63}{{#1}}}
    \newcommand{\VerbatimStringTok}[1]{\textcolor[rgb]{0.25,0.44,0.63}{{#1}}}
    \newcommand{\SpecialStringTok}[1]{\textcolor[rgb]{0.73,0.40,0.53}{{#1}}}
    \newcommand{\ImportTok}[1]{{#1}}
    \newcommand{\DocumentationTok}[1]{\textcolor[rgb]{0.73,0.13,0.13}{\textit{{#1}}}}
    \newcommand{\AnnotationTok}[1]{\textcolor[rgb]{0.38,0.63,0.69}{\textbf{\textit{{#1}}}}}
    \newcommand{\CommentVarTok}[1]{\textcolor[rgb]{0.38,0.63,0.69}{\textbf{\textit{{#1}}}}}
    \newcommand{\VariableTok}[1]{\textcolor[rgb]{0.10,0.09,0.49}{{#1}}}
    \newcommand{\ControlFlowTok}[1]{\textcolor[rgb]{0.00,0.44,0.13}{\textbf{{#1}}}}
    \newcommand{\OperatorTok}[1]{\textcolor[rgb]{0.40,0.40,0.40}{{#1}}}
    \newcommand{\BuiltInTok}[1]{{#1}}
    \newcommand{\ExtensionTok}[1]{{#1}}
    \newcommand{\PreprocessorTok}[1]{\textcolor[rgb]{0.74,0.48,0.00}{{#1}}}
    \newcommand{\AttributeTok}[1]{\textcolor[rgb]{0.49,0.56,0.16}{{#1}}}
    \newcommand{\InformationTok}[1]{\textcolor[rgb]{0.38,0.63,0.69}{\textbf{\textit{{#1}}}}}
    \newcommand{\WarningTok}[1]{\textcolor[rgb]{0.38,0.63,0.69}{\textbf{\textit{{#1}}}}}
    
    
    % Define a nice break command that doesn't care if a line doesn't already
    % exist.
    \def\br{\hspace*{\fill} \\* }
    % Math Jax compatability definitions
    \def\gt{>}
    \def\lt{<}
    % Document parameters
    \title{Harrison-Beard-ProbSet2-Code}
    
    
    

    % Pygments definitions
    
\makeatletter
\def\PY@reset{\let\PY@it=\relax \let\PY@bf=\relax%
    \let\PY@ul=\relax \let\PY@tc=\relax%
    \let\PY@bc=\relax \let\PY@ff=\relax}
\def\PY@tok#1{\csname PY@tok@#1\endcsname}
\def\PY@toks#1+{\ifx\relax#1\empty\else%
    \PY@tok{#1}\expandafter\PY@toks\fi}
\def\PY@do#1{\PY@bc{\PY@tc{\PY@ul{%
    \PY@it{\PY@bf{\PY@ff{#1}}}}}}}
\def\PY#1#2{\PY@reset\PY@toks#1+\relax+\PY@do{#2}}

\expandafter\def\csname PY@tok@w\endcsname{\def\PY@tc##1{\textcolor[rgb]{0.73,0.73,0.73}{##1}}}
\expandafter\def\csname PY@tok@c\endcsname{\let\PY@it=\textit\def\PY@tc##1{\textcolor[rgb]{0.25,0.50,0.50}{##1}}}
\expandafter\def\csname PY@tok@cp\endcsname{\def\PY@tc##1{\textcolor[rgb]{0.74,0.48,0.00}{##1}}}
\expandafter\def\csname PY@tok@k\endcsname{\let\PY@bf=\textbf\def\PY@tc##1{\textcolor[rgb]{0.00,0.50,0.00}{##1}}}
\expandafter\def\csname PY@tok@kp\endcsname{\def\PY@tc##1{\textcolor[rgb]{0.00,0.50,0.00}{##1}}}
\expandafter\def\csname PY@tok@kt\endcsname{\def\PY@tc##1{\textcolor[rgb]{0.69,0.00,0.25}{##1}}}
\expandafter\def\csname PY@tok@o\endcsname{\def\PY@tc##1{\textcolor[rgb]{0.40,0.40,0.40}{##1}}}
\expandafter\def\csname PY@tok@ow\endcsname{\let\PY@bf=\textbf\def\PY@tc##1{\textcolor[rgb]{0.67,0.13,1.00}{##1}}}
\expandafter\def\csname PY@tok@nb\endcsname{\def\PY@tc##1{\textcolor[rgb]{0.00,0.50,0.00}{##1}}}
\expandafter\def\csname PY@tok@nf\endcsname{\def\PY@tc##1{\textcolor[rgb]{0.00,0.00,1.00}{##1}}}
\expandafter\def\csname PY@tok@nc\endcsname{\let\PY@bf=\textbf\def\PY@tc##1{\textcolor[rgb]{0.00,0.00,1.00}{##1}}}
\expandafter\def\csname PY@tok@nn\endcsname{\let\PY@bf=\textbf\def\PY@tc##1{\textcolor[rgb]{0.00,0.00,1.00}{##1}}}
\expandafter\def\csname PY@tok@ne\endcsname{\let\PY@bf=\textbf\def\PY@tc##1{\textcolor[rgb]{0.82,0.25,0.23}{##1}}}
\expandafter\def\csname PY@tok@nv\endcsname{\def\PY@tc##1{\textcolor[rgb]{0.10,0.09,0.49}{##1}}}
\expandafter\def\csname PY@tok@no\endcsname{\def\PY@tc##1{\textcolor[rgb]{0.53,0.00,0.00}{##1}}}
\expandafter\def\csname PY@tok@nl\endcsname{\def\PY@tc##1{\textcolor[rgb]{0.63,0.63,0.00}{##1}}}
\expandafter\def\csname PY@tok@ni\endcsname{\let\PY@bf=\textbf\def\PY@tc##1{\textcolor[rgb]{0.60,0.60,0.60}{##1}}}
\expandafter\def\csname PY@tok@na\endcsname{\def\PY@tc##1{\textcolor[rgb]{0.49,0.56,0.16}{##1}}}
\expandafter\def\csname PY@tok@nt\endcsname{\let\PY@bf=\textbf\def\PY@tc##1{\textcolor[rgb]{0.00,0.50,0.00}{##1}}}
\expandafter\def\csname PY@tok@nd\endcsname{\def\PY@tc##1{\textcolor[rgb]{0.67,0.13,1.00}{##1}}}
\expandafter\def\csname PY@tok@s\endcsname{\def\PY@tc##1{\textcolor[rgb]{0.73,0.13,0.13}{##1}}}
\expandafter\def\csname PY@tok@sd\endcsname{\let\PY@it=\textit\def\PY@tc##1{\textcolor[rgb]{0.73,0.13,0.13}{##1}}}
\expandafter\def\csname PY@tok@si\endcsname{\let\PY@bf=\textbf\def\PY@tc##1{\textcolor[rgb]{0.73,0.40,0.53}{##1}}}
\expandafter\def\csname PY@tok@se\endcsname{\let\PY@bf=\textbf\def\PY@tc##1{\textcolor[rgb]{0.73,0.40,0.13}{##1}}}
\expandafter\def\csname PY@tok@sr\endcsname{\def\PY@tc##1{\textcolor[rgb]{0.73,0.40,0.53}{##1}}}
\expandafter\def\csname PY@tok@ss\endcsname{\def\PY@tc##1{\textcolor[rgb]{0.10,0.09,0.49}{##1}}}
\expandafter\def\csname PY@tok@sx\endcsname{\def\PY@tc##1{\textcolor[rgb]{0.00,0.50,0.00}{##1}}}
\expandafter\def\csname PY@tok@m\endcsname{\def\PY@tc##1{\textcolor[rgb]{0.40,0.40,0.40}{##1}}}
\expandafter\def\csname PY@tok@gh\endcsname{\let\PY@bf=\textbf\def\PY@tc##1{\textcolor[rgb]{0.00,0.00,0.50}{##1}}}
\expandafter\def\csname PY@tok@gu\endcsname{\let\PY@bf=\textbf\def\PY@tc##1{\textcolor[rgb]{0.50,0.00,0.50}{##1}}}
\expandafter\def\csname PY@tok@gd\endcsname{\def\PY@tc##1{\textcolor[rgb]{0.63,0.00,0.00}{##1}}}
\expandafter\def\csname PY@tok@gi\endcsname{\def\PY@tc##1{\textcolor[rgb]{0.00,0.63,0.00}{##1}}}
\expandafter\def\csname PY@tok@gr\endcsname{\def\PY@tc##1{\textcolor[rgb]{1.00,0.00,0.00}{##1}}}
\expandafter\def\csname PY@tok@ge\endcsname{\let\PY@it=\textit}
\expandafter\def\csname PY@tok@gs\endcsname{\let\PY@bf=\textbf}
\expandafter\def\csname PY@tok@gp\endcsname{\let\PY@bf=\textbf\def\PY@tc##1{\textcolor[rgb]{0.00,0.00,0.50}{##1}}}
\expandafter\def\csname PY@tok@go\endcsname{\def\PY@tc##1{\textcolor[rgb]{0.53,0.53,0.53}{##1}}}
\expandafter\def\csname PY@tok@gt\endcsname{\def\PY@tc##1{\textcolor[rgb]{0.00,0.27,0.87}{##1}}}
\expandafter\def\csname PY@tok@err\endcsname{\def\PY@bc##1{\setlength{\fboxsep}{0pt}\fcolorbox[rgb]{1.00,0.00,0.00}{1,1,1}{\strut ##1}}}
\expandafter\def\csname PY@tok@kc\endcsname{\let\PY@bf=\textbf\def\PY@tc##1{\textcolor[rgb]{0.00,0.50,0.00}{##1}}}
\expandafter\def\csname PY@tok@kd\endcsname{\let\PY@bf=\textbf\def\PY@tc##1{\textcolor[rgb]{0.00,0.50,0.00}{##1}}}
\expandafter\def\csname PY@tok@kn\endcsname{\let\PY@bf=\textbf\def\PY@tc##1{\textcolor[rgb]{0.00,0.50,0.00}{##1}}}
\expandafter\def\csname PY@tok@kr\endcsname{\let\PY@bf=\textbf\def\PY@tc##1{\textcolor[rgb]{0.00,0.50,0.00}{##1}}}
\expandafter\def\csname PY@tok@bp\endcsname{\def\PY@tc##1{\textcolor[rgb]{0.00,0.50,0.00}{##1}}}
\expandafter\def\csname PY@tok@fm\endcsname{\def\PY@tc##1{\textcolor[rgb]{0.00,0.00,1.00}{##1}}}
\expandafter\def\csname PY@tok@vc\endcsname{\def\PY@tc##1{\textcolor[rgb]{0.10,0.09,0.49}{##1}}}
\expandafter\def\csname PY@tok@vg\endcsname{\def\PY@tc##1{\textcolor[rgb]{0.10,0.09,0.49}{##1}}}
\expandafter\def\csname PY@tok@vi\endcsname{\def\PY@tc##1{\textcolor[rgb]{0.10,0.09,0.49}{##1}}}
\expandafter\def\csname PY@tok@vm\endcsname{\def\PY@tc##1{\textcolor[rgb]{0.10,0.09,0.49}{##1}}}
\expandafter\def\csname PY@tok@sa\endcsname{\def\PY@tc##1{\textcolor[rgb]{0.73,0.13,0.13}{##1}}}
\expandafter\def\csname PY@tok@sb\endcsname{\def\PY@tc##1{\textcolor[rgb]{0.73,0.13,0.13}{##1}}}
\expandafter\def\csname PY@tok@sc\endcsname{\def\PY@tc##1{\textcolor[rgb]{0.73,0.13,0.13}{##1}}}
\expandafter\def\csname PY@tok@dl\endcsname{\def\PY@tc##1{\textcolor[rgb]{0.73,0.13,0.13}{##1}}}
\expandafter\def\csname PY@tok@s2\endcsname{\def\PY@tc##1{\textcolor[rgb]{0.73,0.13,0.13}{##1}}}
\expandafter\def\csname PY@tok@sh\endcsname{\def\PY@tc##1{\textcolor[rgb]{0.73,0.13,0.13}{##1}}}
\expandafter\def\csname PY@tok@s1\endcsname{\def\PY@tc##1{\textcolor[rgb]{0.73,0.13,0.13}{##1}}}
\expandafter\def\csname PY@tok@mb\endcsname{\def\PY@tc##1{\textcolor[rgb]{0.40,0.40,0.40}{##1}}}
\expandafter\def\csname PY@tok@mf\endcsname{\def\PY@tc##1{\textcolor[rgb]{0.40,0.40,0.40}{##1}}}
\expandafter\def\csname PY@tok@mh\endcsname{\def\PY@tc##1{\textcolor[rgb]{0.40,0.40,0.40}{##1}}}
\expandafter\def\csname PY@tok@mi\endcsname{\def\PY@tc##1{\textcolor[rgb]{0.40,0.40,0.40}{##1}}}
\expandafter\def\csname PY@tok@il\endcsname{\def\PY@tc##1{\textcolor[rgb]{0.40,0.40,0.40}{##1}}}
\expandafter\def\csname PY@tok@mo\endcsname{\def\PY@tc##1{\textcolor[rgb]{0.40,0.40,0.40}{##1}}}
\expandafter\def\csname PY@tok@ch\endcsname{\let\PY@it=\textit\def\PY@tc##1{\textcolor[rgb]{0.25,0.50,0.50}{##1}}}
\expandafter\def\csname PY@tok@cm\endcsname{\let\PY@it=\textit\def\PY@tc##1{\textcolor[rgb]{0.25,0.50,0.50}{##1}}}
\expandafter\def\csname PY@tok@cpf\endcsname{\let\PY@it=\textit\def\PY@tc##1{\textcolor[rgb]{0.25,0.50,0.50}{##1}}}
\expandafter\def\csname PY@tok@c1\endcsname{\let\PY@it=\textit\def\PY@tc##1{\textcolor[rgb]{0.25,0.50,0.50}{##1}}}
\expandafter\def\csname PY@tok@cs\endcsname{\let\PY@it=\textit\def\PY@tc##1{\textcolor[rgb]{0.25,0.50,0.50}{##1}}}

\def\PYZbs{\char`\\}
\def\PYZus{\char`\_}
\def\PYZob{\char`\{}
\def\PYZcb{\char`\}}
\def\PYZca{\char`\^}
\def\PYZam{\char`\&}
\def\PYZlt{\char`\<}
\def\PYZgt{\char`\>}
\def\PYZsh{\char`\#}
\def\PYZpc{\char`\%}
\def\PYZdl{\char`\$}
\def\PYZhy{\char`\-}
\def\PYZsq{\char`\'}
\def\PYZdq{\char`\"}
\def\PYZti{\char`\~}
% for compatibility with earlier versions
\def\PYZat{@}
\def\PYZlb{[}
\def\PYZrb{]}
\makeatother


    % Exact colors from NB
    \definecolor{incolor}{rgb}{0.0, 0.0, 0.5}
    \definecolor{outcolor}{rgb}{0.545, 0.0, 0.0}



    
    % Prevent overflowing lines due to hard-to-break entities
    \sloppy 
    % Setup hyperref package
    \hypersetup{
      breaklinks=true,  % so long urls are correctly broken across lines
      colorlinks=true,
      urlcolor=urlcolor,
      linkcolor=linkcolor,
      citecolor=citecolor,
      }
    % Slightly bigger margins than the latex defaults
    
    \geometry{verbose,tmargin=1in,bmargin=1in,lmargin=1in,rmargin=1in}
    
    

    \begin{document}
    
    
    \maketitle
    
    

    
    \begin{Verbatim}[commandchars=\\\{\}]
{\color{incolor}In [{\color{incolor} }]:} \PY{k+kn}{import} \PY{n+nn}{numpy} \PY{k}{as} \PY{n+nn}{np}
        \PY{k+kn}{from} \PY{n+nn}{numpy}\PY{n+nn}{.}\PY{n+nn}{linalg} \PY{k}{import} \PY{n}{matrix\PYZus{}power}
        \PY{k+kn}{from} \PY{n+nn}{scipy}\PY{n+nn}{.}\PY{n+nn}{stats} \PY{k}{import} \PY{n}{norm}\PY{p}{,} \PY{n}{gaussian\PYZus{}kde}\PY{p}{,} \PY{n}{beta}
        \PY{k+kn}{import} \PY{n+nn}{scipy}\PY{n+nn}{.}\PY{n+nn}{linalg} \PY{k}{as} \PY{n+nn}{la}
        \PY{k+kn}{import} \PY{n+nn}{matplotlib}
        \PY{k+kn}{import} \PY{n+nn}{matplotlib}\PY{n+nn}{.}\PY{n+nn}{pyplot} \PY{k}{as} \PY{n+nn}{plt}
        \PY{k+kn}{from} \PY{n+nn}{numba} \PY{k}{import} \PY{n}{vectorize}\PY{p}{,} \PY{n}{jit}\PY{p}{,} \PY{n}{njit}\PY{p}{,} \PY{n}{float64}\PY{p}{,} \PY{n}{prange}
\end{Verbatim}


    \subsection*{Exercise 1.1.}\label{exercise-1.1.}
\addcontentsline{toc}{subsection}{Exercise 1.1.}

    Let \(X\) be an \(n\times n\) matrix with all positive entries. The
spectral radius \(r(X)\) of \(X\) is the maximum of
\(\left|\lambda\right|\) over all eigenvalues \(\lambda\) of \(X\) where
\(\left|\cdot\right|\) is the modulus of a complex number.

A version of the local spectral radius theorem states that if \(X\) has
all positive entries and \(v\) is any strictly positive \(n\times1\)
vector, then \[
\lim_{i\to\infty}\Vert X^{i}v \Vert^{1/i}\to r(X)\qquad(\mathrm{LSR})
\] where \(\Vert \cdot \Vert\) is the usual Euclidean norm.

Intuitively, the norm of the iterates of a positive vector scale like
\(r(X)\) asymptotically.

The data file \(\mathtt{matrix}\_\mathtt{data.txt}\) contains the data
for a single matrix \(X\).

\begin{enumerate}
\def\labelenumi{\arabic{enumi}.}
\item
  Read it in and compute the spectral radius using the tools for working
  with eigenvalues in \(\mathtt{scipy.linalg}\).
\item
  Test the claim in (LSR) iteratively, computing
  \(\Vert X^{i}v \Vert^{1/i}\) for successively larger values of \(i\).
  See if the sequence so generated converges to \(r(A)\).
\end{enumerate}

    \begin{Verbatim}[commandchars=\\\{\}]
{\color{incolor}In [{\color{incolor}49}]:} \PY{n}{matrix} \PY{o}{=} \PY{n}{np}\PY{o}{.}\PY{n}{loadtxt}\PY{p}{(}\PY{l+s+s2}{\PYZdq{}}\PY{l+s+s2}{matrix\PYZus{}data.txt}\PY{l+s+s2}{\PYZdq{}}\PY{p}{)}
         \PY{n}{λ} \PY{o}{=} \PY{n}{la}\PY{o}{.}\PY{n}{eigvals}\PY{p}{(}\PY{n}{mat}\PY{p}{)}
         \PY{n}{spectral\PYZus{}radius} \PY{o}{=} \PY{n}{np}\PY{o}{.}\PY{n}{absolute}\PY{p}{(}\PY{n}{λ}\PY{p}{)}\PY{o}{.}\PY{n}{max}\PY{p}{(}\PY{p}{)}
\end{Verbatim}


    \begin{Verbatim}[commandchars=\\\{\}]
{\color{incolor}In [{\color{incolor}50}]:} \PY{k}{def} \PY{n+nf}{problem1\PYZus{}1}\PY{p}{(}\PY{n}{v}\PY{p}{,} \PY{n}{X}\PY{p}{,} \PY{n}{i}\PY{p}{)}\PY{p}{:}
             \PY{n}{r} \PY{o}{=} \PY{n}{la}\PY{o}{.}\PY{n}{norm}\PY{p}{(}\PY{n}{matrix\PYZus{}power}\PY{p}{(}\PY{n}{X}\PY{p}{,} \PY{n}{i}\PY{p}{)} \PY{o}{@} \PY{n}{v}\PY{p}{)} \PY{o}{*}\PY{o}{*} \PY{p}{(}\PY{l+m+mi}{1}\PY{o}{/}\PY{n}{i}\PY{p}{)}
             \PY{n}{distance} \PY{o}{=} \PY{n}{np}\PY{o}{.}\PY{n}{absolute}\PY{p}{(}\PY{n}{r} \PY{o}{\PYZhy{}} \PY{n}{spectral\PYZus{}radius}\PY{p}{)}
             \PY{k}{return} \PY{n}{r}\PY{p}{,} \PY{n}{distance}
         
         \PY{n}{j}\PY{p}{,} \PY{n}{jj} \PY{o}{=} \PY{n}{matrix}\PY{o}{.}\PY{n}{shape}
         \PY{n}{v} \PY{o}{=} \PY{n}{np}\PY{o}{.}\PY{n}{ones}\PY{p}{(}\PY{n}{j}\PY{p}{)}
\end{Verbatim}


    \begin{Verbatim}[commandchars=\\\{\}]
{\color{incolor}In [{\color{incolor}51}]:} \PY{n}{i} \PY{o}{=} \PY{l+m+mi}{1000}
         \PY{n}{r}\PY{p}{,} \PY{n}{distance} \PY{o}{=} \PY{n}{problem1\PYZus{}1}\PY{p}{(}\PY{n}{v}\PY{p}{,} \PY{n}{matrix}\PY{p}{,} \PY{n}{i}\PY{p}{)}
         \PY{n+nb}{print}\PY{p}{(}\PY{n}{distance}\PY{p}{)}
\end{Verbatim}


    \begin{Verbatim}[commandchars=\\\{\}]
0.0012011304694321545

    \end{Verbatim}

    \begin{Verbatim}[commandchars=\\\{\}]
{\color{incolor}In [{\color{incolor}52}]:} \PY{n}{i} \PY{o}{=} \PY{l+m+mi}{100000}
         \PY{n}{r}\PY{p}{,} \PY{n}{distance} \PY{o}{=} \PY{n}{problem1\PYZus{}1}\PY{p}{(}\PY{n}{v}\PY{p}{,} \PY{n}{matrix}\PY{p}{,} \PY{n}{i}\PY{p}{)}
         \PY{n+nb}{print}\PY{p}{(}\PY{n}{distance}\PY{p}{)}
\end{Verbatim}


    \begin{Verbatim}[commandchars=\\\{\}]
1.200420037306138e-05

    \end{Verbatim}

    \subsection*{Exercise 1.2.}\label{exercise-1.2.}
\addcontentsline{toc}{subsection}{Exercise 1.2.}

    Recall the the quadratic map generates time series of the form \[
x_{t+1}=4x_{t}(1-x_{t})
\] for some given \(x_{0}\), and that these trajectories are chaotic.

This means that different initial conditions generate seemingly very
different outcomes.

Nevertheless, the regions of the state space where these trajectories
spend most of their time are in fact typically invariant to the initial
condition.

Illustrate this by generating 100 histograms of the time series
generated from the quadratic map, with \(x_{0}\) drawn independently
form the uniform distribution on \((0,1)\).

Do they all look alike?

Try to make your code efficient.

    \begin{Verbatim}[commandchars=\\\{\}]
{\color{incolor}In [{\color{incolor}53}]:} \PY{n+nd}{@njit}
         \PY{k}{def} \PY{n+nf}{problem1\PYZus{}2}\PY{p}{(}\PY{n}{x\PYZus{}0}\PY{p}{,} \PY{n}{length}\PY{p}{)}\PY{p}{:}
             \PY{n}{x} \PY{o}{=} \PY{n}{x\PYZus{}0}
             \PY{k}{for} \PY{n}{t} \PY{o+ow}{in} \PY{n+nb}{range}\PY{p}{(}\PY{n}{length}\PY{o}{\PYZhy{}}\PY{l+m+mi}{1}\PY{p}{)}\PY{p}{:}
                 \PY{n}{x} \PY{o}{=} \PY{l+m+mi}{4}\PY{o}{*}\PY{n}{x}\PY{o}{*}\PY{p}{(}\PY{l+m+mi}{1}\PY{o}{\PYZhy{}}\PY{n}{x}\PY{p}{)}
             \PY{k}{return} \PY{n}{x}
\end{Verbatim}


    \begin{Verbatim}[commandchars=\\\{\}]
{\color{incolor}In [{\color{incolor}54}]:} \PY{n+nb}{print}\PY{p}{(}\PY{n}{problem1\PYZus{}2}\PY{p}{(}\PY{l+m+mf}{0.4}\PY{p}{,}\PY{l+m+mi}{10}\PY{p}{)}\PY{p}{)}
\end{Verbatim}


    \begin{Verbatim}[commandchars=\\\{\}]
0.918969052370147

    \end{Verbatim}

    \begin{Verbatim}[commandchars=\\\{\}]
{\color{incolor}In [{\color{incolor}55}]:} \PY{o}{\PYZpc{}\PYZpc{}}\PY{k}{time}
         length = 10000000
         xknot = np.random.rand(100)
         out = np.empty(xknot.size)
         
         for i,x\PYZus{}0 in enumerate(xknot):
             out[i] = problem1\PYZus{}2(x\PYZus{}0, length = length)
             
         plt.hist(out)
\end{Verbatim}


    \begin{Verbatim}[commandchars=\\\{\}]
CPU times: user 2.04 s, sys: 13.7 ms, total: 2.06 s
Wall time: 2.07 s

    \end{Verbatim}

    \begin{center}
    \adjustimage{max size={0.9\linewidth}{0.9\paperheight}}{output_11_1.png}
    \end{center}
    { \hspace*{\fill} \\}
    
    \subsection*{Exercise 1.3.}\label{exercise-1.3.}
\addcontentsline{toc}{subsection}{Exercise 1.3.}

    In the lecture it was claimed that, if \(\left(\mathbb{X},g\right)\) is
a dynamical system, \(g\) continuous at \(\hat{x}\in\mathbb{X}\), and
for some \(x\in\mathbb{X}\), \(g^{t}(x)\to\hat{x}\), then \(\hat{x}\) is
a steady state of \(\left(\mathbb{X},g\right)\). Prove this.

    \paragraph{Solution.}\label{solution.}
\addcontentsline{toc}{paragraph}{Solution.}

Let
\(y\in C = \left\{ x \in \mathbb{X} \mid \lim_{t\to\infty} g^t (x) = \hat{x} \right\}\).
Since \(g\) is continuous at \(\hat{x}\), and
\(\lim_{t\to\infty} g^{t-1} (y) = \hat{x}\), we have that \[
\begin{eqnarray*}
\hat{x}&=&\lim_{t\to\infty} g^t (x) \\
&=& g\left( \lim_{t\to\infty}g^{t-1}(x)\right) \\
&=&g\left(\hat x\right). 
\end{eqnarray*}
\] \(\blacksquare\)

    \subsection*{Exercise 2.1.}\label{exercise-2.1.}
\addcontentsline{toc}{subsection}{Exercise 2.1.}

    \begin{Verbatim}[commandchars=\\\{\}]
{\color{incolor}In [{\color{incolor}56}]:} \PY{k}{class} \PY{n+nc}{problem2\PYZus{}1}\PY{p}{:}
             
             \PY{k}{def} \PY{n+nf}{\PYZus{}\PYZus{}init\PYZus{}\PYZus{}}\PY{p}{(}\PY{n+nb+bp}{self}\PY{p}{,} \PY{n}{X}\PY{p}{,} \PY{n}{h}\PY{o}{=}\PY{k+kc}{None}\PY{p}{)}\PY{p}{:}
                 
                 \PY{n+nb+bp}{self}\PY{o}{.}\PY{n}{X} \PY{o}{=} \PY{n}{X}
                 \PY{n+nb+bp}{self}\PY{o}{.}\PY{n}{n} \PY{o}{=} \PY{n}{X}\PY{o}{.}\PY{n}{size}
                 
                 \PY{k}{if} \PY{o+ow}{not} \PY{n}{h}\PY{p}{:}
                     \PY{n+nb+bp}{self}\PY{o}{.}\PY{n}{h} \PY{o}{=} \PY{n+nb+bp}{self}\PY{o}{.}\PY{n}{silverman}\PY{p}{(}\PY{p}{)}
                 \PY{k}{else}\PY{p}{:}
                     \PY{n+nb+bp}{self}\PY{o}{.}\PY{n}{h} \PY{o}{=} \PY{n}{h}
                     
             \PY{k}{def} \PY{n+nf}{silverman}\PY{p}{(}\PY{n+nb+bp}{self}\PY{p}{)}\PY{p}{:}
                 \PY{k}{return} \PY{l+m+mf}{1.06} \PY{o}{*} \PY{p}{(}\PY{n+nb+bp}{self}\PY{o}{.}\PY{n}{n} \PY{o}{*}\PY{o}{*} \PY{p}{(}\PY{o}{\PYZhy{}}\PY{l+m+mi}{1}\PY{o}{/}\PY{l+m+mi}{5}\PY{p}{)}\PY{p}{)} \PY{o}{*} \PY{n}{np}\PY{o}{.}\PY{n}{sqrt}\PY{p}{(}\PY{n}{np}\PY{o}{.}\PY{n}{var}\PY{p}{(}\PY{n+nb+bp}{self}\PY{o}{.}\PY{n}{X}\PY{p}{)}\PY{p}{)}
             
             \PY{k}{def} \PY{n+nf}{f}\PY{p}{(}\PY{n+nb+bp}{self}\PY{p}{,} \PY{n}{x}\PY{p}{)}\PY{p}{:}
                 \PY{n}{K} \PY{o}{=} \PY{n}{norm}\PY{o}{.}\PY{n}{pdf}
                 \PY{n}{summand} \PY{o}{=} \PY{n}{K}\PY{p}{(} \PY{p}{(}\PY{n}{x} \PY{o}{\PYZhy{}} \PY{n+nb+bp}{self}\PY{o}{.}\PY{n}{X}\PY{p}{)} \PY{o}{/} \PY{n+nb+bp}{self}\PY{o}{.}\PY{n}{h} \PY{p}{)}
                 \PY{k}{return} \PY{p}{(}\PY{l+m+mi}{1}\PY{o}{/}\PY{p}{(}\PY{n+nb+bp}{self}\PY{o}{.}\PY{n}{h}\PY{o}{*}\PY{n+nb+bp}{self}\PY{o}{.}\PY{n}{n}\PY{p}{)}\PY{p}{)} \PY{o}{*} \PY{n}{summand}\PY{o}{.}\PY{n}{sum}\PY{p}{(}\PY{p}{)}
             
             \PY{k}{def} \PY{n+nf}{estimate}\PY{p}{(}\PY{n+nb+bp}{self}\PY{p}{,} \PY{n}{grid}\PY{o}{=}\PY{n}{np}\PY{o}{.}\PY{n}{linspace}\PY{p}{(}\PY{l+m+mi}{0}\PY{p}{,}\PY{l+m+mi}{1}\PY{p}{,}\PY{l+m+mi}{1000}\PY{p}{)}\PY{p}{)}\PY{p}{:}
                 \PY{n}{estimation} \PY{o}{=} \PY{n}{np}\PY{o}{.}\PY{n}{empty\PYZus{}like}\PY{p}{(}\PY{n}{grid}\PY{p}{)}
                 
                 \PY{k}{for} \PY{n}{j}\PY{p}{,}\PY{n}{k} \PY{o+ow}{in} \PY{n+nb}{enumerate}\PY{p}{(}\PY{n}{grid}\PY{p}{)}\PY{p}{:}
                     \PY{n}{estimation}\PY{p}{[}\PY{n}{j}\PY{p}{]} \PY{o}{=} \PY{n+nb+bp}{self}\PY{o}{.}\PY{n}{f}\PY{p}{(}\PY{n}{k}\PY{p}{)}
                     
                 \PY{k}{return} \PY{n}{estimation}
\end{Verbatim}


    \begin{Verbatim}[commandchars=\\\{\}]
{\color{incolor}In [{\color{incolor}57}]:} \PY{n}{n}\PY{p}{,} \PY{n}{grid} \PY{o}{=} \PY{l+m+mi}{1000}\PY{p}{,} \PY{n}{np}\PY{o}{.}\PY{n}{linspace}\PY{p}{(}\PY{l+m+mi}{0}\PY{p}{,}\PY{l+m+mi}{1}\PY{p}{,}\PY{l+m+mi}{1000}\PY{p}{)}
\end{Verbatim}


    \begin{Verbatim}[commandchars=\\\{\}]
{\color{incolor}In [{\color{incolor}58}]:} \PY{n}{α2β2} \PY{o}{=} \PY{n}{np}\PY{o}{.}\PY{n}{random}\PY{o}{.}\PY{n}{beta}\PY{p}{(}\PY{l+m+mi}{2}\PY{p}{,}\PY{l+m+mi}{2}\PY{p}{,}\PY{n}{size}\PY{o}{=}\PY{p}{(}\PY{n}{n}\PY{p}{,}\PY{l+m+mi}{1}\PY{p}{)}\PY{p}{)}
         \PY{n}{estimator\PYZus{}α2β2} \PY{o}{=} \PY{n}{problem2\PYZus{}1}\PY{p}{(}\PY{n}{X} \PY{o}{=} \PY{n}{α2β2}\PY{p}{)}
         \PY{n}{estimate\PYZus{}α2β2} \PY{o}{=} \PY{n}{estimator\PYZus{}α2β2}\PY{o}{.}\PY{n}{estimate}\PY{p}{(}\PY{n}{grid}\PY{o}{=}\PY{n}{grid}\PY{p}{)}
         \PY{n}{plt}\PY{o}{.}\PY{n}{plot}\PY{p}{(}\PY{n}{grid}\PY{p}{,} \PY{n}{estimate\PYZus{}α2β2}\PY{p}{,} \PY{n}{label} \PY{o}{=} \PY{l+s+s2}{\PYZdq{}}\PY{l+s+s2}{Estimated}\PY{l+s+s2}{\PYZdq{}}\PY{p}{)}
         \PY{n}{plt}\PY{o}{.}\PY{n}{plot}\PY{p}{(}\PY{n}{grid}\PY{p}{,} \PY{n}{beta}\PY{o}{.}\PY{n}{pdf}\PY{p}{(}\PY{n}{grid}\PY{p}{,}\PY{l+m+mi}{2}\PY{p}{,}\PY{l+m+mi}{2}\PY{p}{)}\PY{p}{,}\PY{n}{label} \PY{o}{=} \PY{l+s+s2}{\PYZdq{}}\PY{l+s+s2}{Actual}\PY{l+s+s2}{\PYZdq{}}\PY{p}{)}
         \PY{n}{plt}\PY{o}{.}\PY{n}{legend}\PY{p}{(}\PY{p}{)}
         \PY{n}{plt}\PY{o}{.}\PY{n}{show}\PY{p}{(}\PY{p}{)}
\end{Verbatim}


    \begin{center}
    \adjustimage{max size={0.9\linewidth}{0.9\paperheight}}{output_18_0.png}
    \end{center}
    { \hspace*{\fill} \\}
    
    \begin{Verbatim}[commandchars=\\\{\}]
{\color{incolor}In [{\color{incolor}59}]:} \PY{n}{α2β5} \PY{o}{=} \PY{n}{np}\PY{o}{.}\PY{n}{random}\PY{o}{.}\PY{n}{beta}\PY{p}{(}\PY{l+m+mi}{2}\PY{p}{,}\PY{l+m+mi}{5}\PY{p}{,}\PY{n}{size}\PY{o}{=}\PY{p}{(}\PY{n}{n}\PY{p}{,}\PY{l+m+mi}{1}\PY{p}{)}\PY{p}{)}
         \PY{n}{estimator\PYZus{}α2β5} \PY{o}{=} \PY{n}{problem2\PYZus{}1}\PY{p}{(}\PY{n}{X}\PY{o}{=}\PY{n}{α2β5}\PY{p}{)}
         \PY{n}{estimate\PYZus{}α2β5} \PY{o}{=} \PY{n}{estimator\PYZus{}α2β5}\PY{o}{.}\PY{n}{estimate}\PY{p}{(}\PY{n}{grid}\PY{o}{=}\PY{n}{grid}\PY{p}{)}
         \PY{n}{plt}\PY{o}{.}\PY{n}{plot}\PY{p}{(}\PY{n}{grid}\PY{p}{,} \PY{n}{estimate\PYZus{}α2β5}\PY{p}{,} \PY{n}{label}\PY{o}{=}\PY{l+s+s2}{\PYZdq{}}\PY{l+s+s2}{Estimated}\PY{l+s+s2}{\PYZdq{}}\PY{p}{)}
         \PY{n}{plt}\PY{o}{.}\PY{n}{plot}\PY{p}{(}\PY{n}{grid}\PY{p}{,} \PY{n}{beta}\PY{o}{.}\PY{n}{pdf}\PY{p}{(}\PY{n}{grid}\PY{p}{,}\PY{l+m+mi}{2}\PY{p}{,}\PY{l+m+mi}{5}\PY{p}{)}\PY{p}{,} \PY{n}{label}\PY{o}{=} \PY{l+s+s2}{\PYZdq{}}\PY{l+s+s2}{Actual}\PY{l+s+s2}{\PYZdq{}}\PY{p}{)}
         \PY{n}{plt}\PY{o}{.}\PY{n}{legend}\PY{p}{(}\PY{p}{)}
         \PY{n}{plt}\PY{o}{.}\PY{n}{show}\PY{p}{(}\PY{p}{)}
\end{Verbatim}


    \begin{center}
    \adjustimage{max size={0.9\linewidth}{0.9\paperheight}}{output_19_0.png}
    \end{center}
    { \hspace*{\fill} \\}
    
    \begin{Verbatim}[commandchars=\\\{\}]
{\color{incolor}In [{\color{incolor}60}]:} \PY{n}{α05β05} \PY{o}{=} \PY{n}{np}\PY{o}{.}\PY{n}{random}\PY{o}{.}\PY{n}{beta}\PY{p}{(}\PY{l+m+mf}{0.5}\PY{p}{,}\PY{l+m+mf}{0.5}\PY{p}{,}\PY{n}{size}\PY{o}{=}\PY{p}{(}\PY{n}{n}\PY{p}{,}\PY{l+m+mi}{1}\PY{p}{)}\PY{p}{)}
         \PY{n}{estimator\PYZus{}α05β05} \PY{o}{=} \PY{n}{problem2\PYZus{}1}\PY{p}{(}\PY{n}{X}\PY{o}{=}\PY{n}{α05β05}\PY{p}{)}
         \PY{n}{estimate\PYZus{}α05β05} \PY{o}{=} \PY{n}{estimator\PYZus{}α05β05}\PY{o}{.}\PY{n}{estimate}\PY{p}{(}\PY{n}{grid}\PY{o}{=}\PY{n}{grid}\PY{p}{)}
         \PY{n}{plt}\PY{o}{.}\PY{n}{plot}\PY{p}{(}\PY{n}{grid}\PY{p}{,} \PY{n}{estimate\PYZus{}α05β05}\PY{p}{,} \PY{n}{label}\PY{o}{=}\PY{l+s+s2}{\PYZdq{}}\PY{l+s+s2}{Estimated}\PY{l+s+s2}{\PYZdq{}}\PY{p}{)}
         \PY{n}{plt}\PY{o}{.}\PY{n}{plot}\PY{p}{(}\PY{n}{grid}\PY{p}{,} \PY{n}{beta}\PY{o}{.}\PY{n}{pdf}\PY{p}{(}\PY{n}{grid}\PY{p}{,}\PY{l+m+mf}{0.5}\PY{p}{,}\PY{l+m+mf}{0.5}\PY{p}{)}\PY{p}{,} \PY{n}{label}\PY{o}{=}\PY{l+s+s2}{\PYZdq{}}\PY{l+s+s2}{Actual}\PY{l+s+s2}{\PYZdq{}}\PY{p}{)}
         \PY{n}{plt}\PY{o}{.}\PY{n}{legend}\PY{p}{(}\PY{p}{)}
         \PY{n}{plt}\PY{o}{.}\PY{n}{show}\PY{p}{(}\PY{p}{)}
\end{Verbatim}


    \begin{center}
    \adjustimage{max size={0.9\linewidth}{0.9\paperheight}}{output_20_0.png}
    \end{center}
    { \hspace*{\fill} \\}
    
    \subsection*{Exercise 2.2.}\label{exercise-2.2.}
\addcontentsline{toc}{subsection}{Exercise 2.2.}

    \begin{Verbatim}[commandchars=\\\{\}]
{\color{incolor}In [{\color{incolor}61}]:} \PY{n}{ρ}\PY{p}{,} \PY{n}{b}\PY{p}{,} \PY{n}{σ}\PY{p}{,} \PY{n}{μ}\PY{p}{,} \PY{n}{s} \PY{o}{=} \PY{l+m+mf}{0.9}\PY{p}{,} \PY{l+m+mf}{0.0}\PY{p}{,} \PY{l+m+mf}{0.1}\PY{p}{,} \PY{o}{\PYZhy{}}\PY{l+m+mi}{3}\PY{p}{,} \PY{l+m+mf}{0.2}
\end{Verbatim}


    \begin{Verbatim}[commandchars=\\\{\}]
{\color{incolor}In [{\color{incolor}62}]:} \PY{k}{def} \PY{n+nf}{problem2\PYZus{}2}\PY{p}{(}\PY{n}{x}\PY{p}{,} \PY{n}{ρ}\PY{o}{=}\PY{n}{ρ}\PY{p}{,} \PY{n}{b}\PY{o}{=}\PY{n}{b}\PY{p}{,} \PY{n}{σ}\PY{o}{=}\PY{n}{σ}\PY{p}{)}\PY{p}{:}
             \PY{n}{ξ} \PY{o}{=} \PY{n}{np}\PY{o}{.}\PY{n}{random}\PY{o}{.}\PY{n}{standard\PYZus{}normal}\PY{p}{(}\PY{p}{)}
             \PY{k}{return} \PY{p}{(}\PY{n}{ρ} \PY{o}{*} \PY{n}{x}\PY{p}{)} \PY{o}{+} \PY{n}{b} \PY{o}{+} \PY{p}{(}\PY{n}{σ} \PY{o}{*} \PY{n}{ξ}\PY{p}{)}
\end{Verbatim}


    \begin{Verbatim}[commandchars=\\\{\}]
{\color{incolor}In [{\color{incolor}66}]:} \PY{n}{grid} \PY{o}{=} \PY{n}{np}\PY{o}{.}\PY{n}{linspace}\PY{p}{(}\PY{l+m+mi}{0}\PY{p}{,}\PY{l+m+mi}{1}\PY{p}{,}\PY{l+m+mi}{1000}\PY{p}{)}
         \PY{n}{ψ} \PY{o}{=} \PY{n}{norm}\PY{o}{.}\PY{n}{pdf}\PY{p}{(}\PY{n}{grid}\PY{p}{,} \PY{n}{loc}\PY{o}{=}\PY{n}{μ}\PY{p}{,} \PY{n}{scale}\PY{o}{=}\PY{n}{s}\PY{o}{*}\PY{o}{*}\PY{l+m+mi}{2}\PY{p}{)}
         \PY{n}{plt}\PY{o}{.}\PY{n}{plot}\PY{p}{(}\PY{n}{np}\PY{o}{.}\PY{n}{linspace}\PY{p}{(}\PY{n}{μ}\PY{o}{\PYZhy{}} \PY{l+m+mi}{3}\PY{o}{*}\PY{n}{s}\PY{p}{,} \PY{n}{μ}\PY{o}{+}\PY{l+m+mi}{3}\PY{o}{*}\PY{n}{s}\PY{p}{)}\PY{p}{,} \PY{n}{matplotlib}\PY{o}{.}\PY{n}{mlab}\PY{o}{.}\PY{n}{normpdf}\PY{p}{(}\PY{n}{np}\PY{o}{.}\PY{n}{linspace}\PY{p}{(}\PY{n}{μ}\PY{o}{\PYZhy{}} \PY{l+m+mi}{3}\PY{o}{*}\PY{n}{s}\PY{p}{,} \PY{n}{μ}\PY{o}{+}\PY{l+m+mi}{3}\PY{o}{*}\PY{n}{s}\PY{p}{)}\PY{p}{,} \PY{n}{μ}\PY{p}{,} \PY{n}{s}\PY{p}{)}\PY{p}{)}
         \PY{n}{plt}\PY{o}{.}\PY{n}{plot}\PY{p}{(}\PY{n}{np}\PY{o}{.}\PY{n}{linspace}\PY{p}{(}\PY{n}{ρ}\PY{o}{*}\PY{n}{μ} \PY{o}{+} \PY{n}{b} \PY{o}{\PYZhy{}} \PY{l+m+mi}{3}\PY{o}{*}\PY{n}{s}\PY{p}{,} \PY{n}{ρ}\PY{o}{*}\PY{n}{μ} \PY{o}{+} \PY{n}{b} \PY{o}{+} \PY{l+m+mi}{3}\PY{o}{*}\PY{n}{s}\PY{p}{)}\PY{p}{,} \PY{n}{matplotlib}\PY{o}{.}\PY{n}{mlab}\PY{o}{.}\PY{n}{normpdf}\PY{p}{(}\PY{n}{np}\PY{o}{.}\PY{n}{linspace}\PY{p}{(}\PY{n}{ρ}\PY{o}{*}\PY{n}{μ} \PY{o}{+} \PY{n}{b} \PY{o}{\PYZhy{}} \PY{l+m+mi}{3}\PY{o}{*}\PY{n}{s}\PY{p}{,} \PY{n}{ρ}\PY{o}{*}\PY{n}{μ} \PY{o}{+} \PY{n}{b} \PY{o}{+} \PY{l+m+mi}{3}\PY{o}{*}\PY{n}{s}\PY{p}{)}\PY{p}{,} \PY{n}{ρ}\PY{o}{*}\PY{n}{μ} \PY{o}{+} \PY{n}{b}\PY{p}{,} \PY{n}{np}\PY{o}{.}\PY{n}{sqrt}\PY{p}{(}\PY{p}{(}\PY{n}{ρ}\PY{o}{*}\PY{o}{*}\PY{l+m+mi}{2}\PY{p}{)}\PY{o}{*}\PY{p}{(}\PY{n}{s}\PY{o}{*}\PY{o}{*}\PY{l+m+mi}{2}\PY{p}{)} \PY{o}{+} \PY{p}{(}\PY{n}{σ}\PY{o}{*}\PY{o}{*}\PY{l+m+mi}{2}\PY{p}{)}\PY{p}{)}\PY{p}{)}\PY{p}{)}
\end{Verbatim}


    \begin{Verbatim}[commandchars=\\\{\}]
/anaconda3/lib/python3.6/site-packages/ipykernel\_launcher.py:3: MatplotlibDeprecationWarning: scipy.stats.norm.pdf
  This is separate from the ipykernel package so we can avoid doing imports until
/anaconda3/lib/python3.6/site-packages/ipykernel\_launcher.py:4: MatplotlibDeprecationWarning: scipy.stats.norm.pdf
  after removing the cwd from sys.path.

    \end{Verbatim}

\begin{Verbatim}[commandchars=\\\{\}]
{\color{outcolor}Out[{\color{outcolor}66}]:} [<matplotlib.lines.Line2D at 0x1a16114278>]
\end{Verbatim}
            
    \begin{center}
    \adjustimage{max size={0.9\linewidth}{0.9\paperheight}}{output_24_2.png}
    \end{center}
    { \hspace*{\fill} \\}
    
    \begin{Verbatim}[commandchars=\\\{\}]
{\color{incolor}In [{\color{incolor}67}]:} \PY{n}{ngrid} \PY{o}{=} \PY{p}{[}\PY{l+m+mi}{10}\PY{p}{,}\PY{l+m+mi}{100}\PY{p}{,}\PY{l+m+mi}{1000}\PY{p}{,}\PY{l+m+mi}{10000}\PY{p}{]}
         \PY{k}{for} \PY{n}{n} \PY{o+ow}{in} \PY{n}{ngrid}\PY{p}{:}
             \PY{n}{dist} \PY{o}{=} \PY{n}{problem2\PYZus{}2}\PY{p}{(}\PY{n}{np}\PY{o}{.}\PY{n}{random}\PY{o}{.}\PY{n}{normal}\PY{p}{(}\PY{n}{μ}\PY{p}{,} \PY{n}{s}\PY{o}{*}\PY{o}{*}\PY{l+m+mi}{2}\PY{p}{,} \PY{n}{n}\PY{p}{)}\PY{p}{)}
             \PY{n}{estimator} \PY{o}{=} \PY{n}{problem2\PYZus{}1}\PY{p}{(}\PY{n}{dist}\PY{p}{)}
             
             \PY{n}{grid} \PY{o}{=} \PY{n}{np}\PY{o}{.}\PY{n}{linspace}\PY{p}{(}\PY{o}{\PYZhy{}}\PY{l+m+mf}{3.2}\PY{p}{,}\PY{o}{\PYZhy{}}\PY{l+m+mf}{2.4}\PY{p}{,}\PY{n}{n}\PY{p}{)}
             \PY{n}{density} \PY{o}{=} \PY{n}{estimator}\PY{o}{.}\PY{n}{estimate}\PY{p}{(}\PY{n}{grid}\PY{o}{=}\PY{n}{grid}\PY{p}{)}
             \PY{n}{plt}\PY{o}{.}\PY{n}{plot}\PY{p}{(}\PY{n}{grid}\PY{p}{,}\PY{n}{density}\PY{p}{,}\PY{n}{label}\PY{o}{=}\PY{l+s+s2}{\PYZdq{}}\PY{l+s+s2}{precision level: }\PY{l+s+si}{\PYZob{}\PYZcb{}}\PY{l+s+s2}{\PYZdq{}}\PY{o}{.}\PY{n}{format}\PY{p}{(}\PY{n}{n}\PY{p}{)}\PY{p}{)}
             
         \PY{n}{plt}\PY{o}{.}\PY{n}{legend}\PY{p}{(}\PY{p}{)}
\end{Verbatim}


\begin{Verbatim}[commandchars=\\\{\}]
{\color{outcolor}Out[{\color{outcolor}67}]:} <matplotlib.legend.Legend at 0x1a238b6748>
\end{Verbatim}
            
    \begin{center}
    \adjustimage{max size={0.9\linewidth}{0.9\paperheight}}{output_25_1.png}
    \end{center}
    { \hspace*{\fill} \\}
    
    \subsection*{Exercise 2.3.}\label{exercise-2.3.}
\addcontentsline{toc}{subsection}{Exercise 2.3.}

    In the lecture it was claimed that, for \(n\times n\) matrix \(A\), we
have \[
r(A)<1\implies A^{k}\to0
\] where convergence is in terms of the spectral norm.

Prove this using Gelfand's formula.

    \paragraph{Solution.}\label{solution.}
\addcontentsline{toc}{paragraph}{Solution.}

Let \(A\in M_{n\times n}\), and let \(A\) have spectral radius
\(r(A)<1\). Consider arbitrary \(\varepsilon\) such that
\(0<\varepsilon<1-r(A)\).

By Gelfand's formula, \(\exists K(\varepsilon)\in\mathbb{N}\) such that
\(\forall k\geq K(\varepsilon)\), we have that \[
\begin{eqnarray*}
(r(A)-\varepsilon)^{k} &<& \Vert A^{k} \Vert \\
&<& (r(A)+\varepsilon)^{k}.
\end{eqnarray*}
\] But \(\Vert A^{k} \Vert \geq 0\), so \[
\begin{eqnarray*}
0 &<& \Vert A^{k} \Vert \\
&<& (r(A)+\varepsilon)^{k}.
\end{eqnarray*}
\] Since \(\varepsilon\) arbitrary, it follows that
\(\lim_{k\to\infty} (r(A)+\varepsilon)^{k}=0\). Thus,
\(\lim_{k\to\infty} \Vert A^k \Vert = 0\).

\(\blacksquare\)


    % Add a bibliography block to the postdoc
    
    
    
    \end{document}
