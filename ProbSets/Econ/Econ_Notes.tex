%% LyX 2.1.4 created this file.  For more info, see http://www.lyx.org/.
%% Do not edit unless you really know what you are doing.
\documentclass[english]{article}
\usepackage[latin9]{inputenc}
\usepackage{geometry}
\geometry{verbose,tmargin=1in,bmargin=1in,lmargin=1in,rmargin=1in}
\usepackage{color}
\definecolor{shadecolor}{rgb}{0.925781, 0.925781, 0.925781}
\usepackage{calc}
\usepackage{framed}
\usepackage{mathtools}
\usepackage{amsmath}
\usepackage{amssymb}
\usepackage{cancel}
\PassOptionsToPackage{normalem}{ulem}
\usepackage{ulem}

\makeatletter
%%%%%%%%%%%%%%%%%%%%%%%%%%%%%% User specified LaTeX commands.
\usepackage{listings}
\usepackage{color}

\definecolor{dkgreen}{rgb}{0,0.6,0}
\definecolor{gray}{rgb}{0.5,0.5,0.5}
\definecolor{mauve}{rgb}{0.58,0,0.82}

\lstset{frame=tb,
  language=Python,
  aboveskip=3mm,
  belowskip=3mm,
  showstringspaces=false,
  columns=flexible,
  basicstyle={\small\ttfamily},
  numbers=none,
  numberstyle=\tiny\color{gray},
  keywordstyle=\color{blue},
  commentstyle=\color{dkgreen},
  stringstyle=\color{mauve},
  breaklines=true,
  breakatwhitespace=true,
  tabsize=3
}



\usepackage{amsmath}

\renewcommand{\familydefault}{\sfdefault}

\usepackage{fancyhdr}
\pagestyle{fancy}


\usepackage{enumitem}
\setlist{nolistsep}
\usepackage{graphicx}



%\usepackage[proportional,scaled=1.064]{erewhon}
%\usepackage[erewhon,vvarbb,bigdelims]{newtxmath}
%\usepackage[T1]{fontenc}
%\renewcommand*\oldstylenums[1]{\textosf{#1}}

\usepackage{tikz}
 
\newcommand*\mycirc[1]{%
   \begin{tikzpicture}
     \node[draw,circle,inner sep=1pt] {#1};
   \end{tikzpicture}}


\usepackage{scalerel,stackengine}
\stackMath
\newcommand\hatt[1]{%
\savestack{\tmpbox}{\stretchto{%
  \scaleto{%
    \scalerel*[\widthof{\ensuremath{#1}}]{\kern.1pt\mathchar"0362\kern.1pt}%
    {\rule{0ex}{\textheight}}%WIDTH-LIMITED CIRCUMFLEX
  }{\textheight}% 
}{2.4ex}}%
\stackon[-6.9pt]{#1}{\tmpbox}%
}
\parskip 1ex





\stackMath
\newcommand\tildee[1]{%
\savestack{\tmpbox}{\stretchto{%
  \scaleto{%
    \scalerel*[\widthof{\ensuremath{#1}}]{\kern.1pt\mathchar"307E\kern.1pt}%
    {\rule{0ex}{\textheight}}%WIDTH-LIMITED CIRCUMFLEX
  }{\textheight}% 
}{2.4ex}}%
\stackon[-6.9pt]{#1}{\tmpbox}%
}
\parskip 1ex










\newcommand{\code}[1]{\texttt{#1}}





\usepackage{tcolorbox}
\tcbuselibrary{theorems}


\newtcbtheorem[]{kb}{Key Concept}%
{colback=blue!10!white,colframe=blue!65!black,fonttitle=\bfseries}{th}




\usepackage{lastpage}


\lhead{Harrison Beard}
\rhead{OSM Boot Camp \textbf{Econ Notes}}
\cfoot{Page  \thepage /\pageref{LastPage}}

\AtBeginDocument{
  \def\labelitemiii{ }
  \def\labelitemiv{ }
}

\makeatother

\usepackage{babel}
\begin{document}
\global\long\def\n#1{\left\Vert #1\right\Vert }
\global\long\def\eval#1{\left.#1\right|}
\global\long\def\R{\mathbb{R}}
\global\long\def\N{\mathbb{N}}
\global\long\def\Quo{\mathbb{Q}}
\global\long\def\F{\mathbb{F}}
\global\long\def\cm{^{\complement}}
\global\long\def\pow#1{\mathcal{P}\left(#1\right)}
\global\long\def\es{\mbox{\ensuremath{\emptyset}}}
\global\long\def\pr{^{\prime}}
\global\long\def\Com{\mathbb{C}}
\global\long\def\part#1#2{\frac{\partial#1}{\partial#2}}
\global\long\def\sm{\smallsetminus}
\global\long\def\usub#1#2#3#4{\underset{\phantom{#3}#2\phantom{#4}}{#3\underbrace{#1}#4}}
\global\long\def\E#1{\mathrm{E}\left[#1\right]}
\global\long\def\cE#1#2{\mathrm{E}\left[\left.#1\phantom{\mathllap{#2}}\right|#2\right]}
\global\long\def\Var#1{\mathrm{Var}\left[#1\right]}
\global\long\def\e#1{\mathrm{e}^{#1}}
\global\long\def\G#1{\Gamma\left(#1\right)}
\global\long\def\ep{\varepsilon}
\global\long\def\P{\mathrm{P} }
\global\long\def\CS#1#2{\left\{  \left.#1\phantom{\mathllap{#2}}\right|#2\right\}  }
\global\long\def\inn#1#2{\left\langle #1,#2\right\rangle }
\global\long\def\span#1{\mathrm{span}\left\{  #1\right\}  }
\global\long\def\H{^{\mathrm{\mathsf{H}}}}
\global\long\def\T{^{\mathsf{T}}}
\global\long\def\tr#1{\mathrm{tr}\left(#1\right)}
\global\long\def\proj#1#2{\mathrm{proj}_{#1}\left(#2\right)}
\global\long\def\d{\mathrm{d}}
\global\long\def\qed{\ \hfill\blacksquare}
\global\long\def\i#1#2{\varint#1\,\mathrm{d}#2}
\global\long\def\diff#1#2{\frac{\mathrm{d}#1}{\mathrm{d}#2}}
\global\long\def\Cov#1#2{\mathrm{Cov}\left[#1,#2\right]}
\global\long\def\nb#1#2{\left\Vert #1\right\Vert _{#2}}
\global\long\def\Fs{\mathrm{F}}
\global\long\def\iid{\stackrel{\mbox{iid}}{\sim}}
\global\long\def\L{\mathscr{L}}
\global\long\def\Norm#1#2{\mathcal{N}\left(#1,#2\right)}
\global\long\def\cCov#1#2#3{\mathrm{Cov}\left[\left.#1,#2\phantom{\mathllap{#3}}\right|#3\right]}
\global\long\def\s{^{\ast}}
\global\long\def\im{\mathrm{im}}
\global\long\def\Skew#1#2{\mathrm{Skew}_{#1}\left(#2\right)}
\global\long\def\rank#1{\mathrm{rank}\left(#1\right)}
\global\long\def\io{\varint}
\global\long\def\Sym#1#2{\mathrm{Sym}_{#1}\left(#2\right)}
\global\long\def\v{\mathbf{v}}
\global\long\def\basis#1{\mathrm{basis}\left(#1\right)}
\global\long\def\l#1{\left(\textit{#1}\right).}
\global\long\def\conv#1{\mathrm{conv}\left(#1\right)}
\global\long\def\x{\mathbf{x}}
\global\long\def\lcr#1#2#3{#1\hfill#2\hfill#3}
\global\long\def\D{\mathbf{D}}
\global\long\def\A{\mathbf{A}}
\global\long\def\B{\mathbf{B}}
\global\long\def\ppr{^{\prime\prime}}
\global\long\def\pppr{^{\prime\prime\prime}}
\global\long\def\ppppr{^{\imath v}}
\global\long\def\u{\mathbf{u}}
\global\long\def\y{\mathbf{y}}
\global\long\def\p{\mathbf{p}}
\global\long\def\z{\mathbf{z}}
\global\long\def\o{\mathbf{0}}
\global\long\def\a{\mathbf{a}}
\global\long\def\b{\mathbf{b}}
\global\long\def\t{\T}
\global\long\def\h{\H}
\global\long\def\r{\mathbf{r}}
\global\long\def\M#1#2{\mathrm{M}_{#1}\left(#2\right)}
\global\long\def\gmm#1{\hat{#1}_{\mathrm{GMM}}}
\global\long\def\mle#1{\hat{#1}_{\mathrm{MLE}}}
\global\long\def\lik#1#2{\mathcal{L}\left(#1\mid#2\right)}
\global\long\def\cs#1#2{\left(#1\mid#2\right)}
\global\long\def\W{\mathbf{W}}
\global\long\def\th{\boldsymbol{\theta}}
\global\long\def\smm#1{\hat{#1}_{\mathrm{SMM}}}
\global\long\def\Unif#1#2{\mathrm{Unif}\left(#1,#2\right)}
\global\long\def\thm{{\color{cyan}\vartriangleright\mbox{ \textbf{Thm. }}}}
\global\long\def\defn{{\color{red}\triangle\mbox{ \textbf{Def. }}}}
\global\long\def\ex{\mbox{\ensuremath{\lozenge}\ \textbf{Example. }}}
\global\long\def\note{\mycirc{!}\,\textbf{Note.}\,}
\global\long\def\lemm{{\color{cyan}\vartriangleright\mbox{ \textbf{Lemma. }}}}
\global\long\def\coro{{\color{cyan}\vartriangleright\mbox{ \textbf{Cor. }}}}
\global\long\def\pf{\square\,\textbf{Proof.}\,}
\global\long\def\c{\mathbf{c}}
\global\long\def\kw#1{\textbf{{\color{blue}#1}}\index{#1}}
\global\long\def\Q{\mathbf{Q}}
\global\long\def\kww#1{\textbf{{\color{white}#1}}\index{#1}}
\global\long\def\break{\smallskip{}}
\global\long\def\bbreak{\bigskip{}\bigskip{}}
\global\long\def\endex{\;\hfill\blacklozenge}
\global\long\def\prop{{\color{cyan}\vartriangleright\mbox{ \textbf{Prop. }}}}
\global\long\def\npg{\newpage{}}
\global\long\def\topc{\mathbf{Topic.}\ }
\global\long\def\dd{\mathbf{d}}
\global\long\def\nn{^{-1}}
\global\long\def\I{\mathbf{I}}
\global\long\def\uv#1{\mathbf{e}_{#1}}


\global\long\def\begday#1#2#3#4{\begin{array}{c}
 \resizebox{5cm}{!}{\textbf{#1, #2 #3. #4}}\qquad\qquad\qquad\,\qquad\qquad\qquad\qquad\qquad\qquad\qquad\qquad\;\;\;\;\;\;\;\qquad\qquad\qquad\qquad\qquad\qquad\qquad\qquad\qquad\qquad\end{array}}



\title{\textsf{OSM Boot Camp }\textsf{\textbf{Econ Notes}}}


\author{\textsf{Harrison Beard}}


\date{\textsf{Summer 2018}}
\maketitle
\begin{itemize}
\item \textbf{$\note$}Recall
\[
\mle{\theta}=\theta:\max_{\theta}\log\lik{\x}{\theta}
\]
\begin{eqnarray*}
\gmm{\theta} & = & \theta:\min_{\theta}\n{m(\x\mid\theta)-m(\x)}\\
 & = & \arg\min_{\th}e(\x\mid\theta)\t\mathbf{W}e(\x\mid\theta).
\end{eqnarray*}
where we can set 
\[
m_{1}(\x\mid\mu,\sigma)=\begin{pmatrix}\E{\x\mid\mu,\sigma}\\
\Var{\x\mid\mu,\sigma}
\end{pmatrix}
\]
 and 
\[
m_{2}\left(\x\mid\mu,\sigma\right)
\]

\item \textbf{$\topc$}OLS. 

\begin{itemize}
\item Consider
\[
y_{i}=\beta_{0}+\beta_{1}x_{1i}+\beta_{2}x_{2i}+\ep_{i},
\]
where
\[
\E{\ep_{i}}=0
\]
and
\[
\E{x_{ji}\ep_{i}}=0.
\]
 
\item Let
\[
\gmm{\theta}=\theta:\min_{\theta}\ep\t\ep.
\]
This is just OLS regression, in the form of GMM.
\item Now consider 
\[
\E{x_{1i}\ep_{i}}=0
\]
and 
\[
\E{x_{2i}\ep_{i}}=0.
\]

\item Computing a moment:
\[
\ep_{i}=y_{i}-\beta_{0}-\beta_{1}x_{1i}-\beta_{2}x_{2i}.
\]
Each error is a function of the data.
\item Let
\begin{eqnarray*}
m_{1}\cs{\x}{\beta_{0},\beta_{1},\beta_{2}} & = & \frac{1}{N}\sum_{i=1}^{N}\\
 & = & \frac{1}{N}\sum_{i}\left(y_{i}-\beta_{0}-\beta_{1}x_{1i}-\beta_{2}x_{2i}\right)\\
 & = & 0.
\end{eqnarray*}
MLE says that choose a set of params so that that sum adds up to zero.
\begin{eqnarray*}
m_{2}\cs{\x}{\beta_{0},\beta_{1},\beta_{2}} & = & \frac{1}{N}\sum_{i=1}^{N}x_{1i}\left(y_{i}-\beta_{0}-\beta_{1}x_{1i}-\beta_{2}x_{2i}\right)\\
 & = & 0.
\end{eqnarray*}
This is another moment condition.
\[
m_{3}\left(\cdot\right)=\cdots x_{2i}\cdots
\]
and so on.
\end{itemize}
\item $\topc$Brock-Mirman model.
\[
1=\frac{\beta\E{r_{t+1}u\pr\left(c_{t+1}\right)}}{u\pr\left(c_{t}\right)}
\]
\[
\implies\beta\E{\frac{r_{t+1}c_{t}}{c_{t+1}}}-1=0.
\]
Middle term is
\[
\beta\E{\frac{\alpha\e{z_{t+1}}k_{t+1}^{\alpha-1}c_{t}}{c_{t+1}}}-1=0.
\]
This is equation (9) in the notes.
\item $\topc$

\begin{itemize}
\item Consider the difference between 
\[
\mle{\th}=\arg\max\th\log\lik{\x}{\th}
\]
and
\[
\gmm{\th}=\arg\min_{\th}\n{\usub{m\cs{\x}{\th}}{\mbox{model}}{}{}-\usub{m(\x)}{\mbox{data}}{}{}}
\]

\end{itemize}
\item $\topc$

\begin{itemize}
\item Let
\[
S\coloneqq\mbox{\# of sims }(s)
\]
\[
\tilde{\x}\coloneqq\left\{ \tilde{\x}_{1},\ldots,\tilde{\x}_{S}\right\} \implies\tilde{\x}_{s}=\begin{pmatrix}y_{1s}x_{11s}x_{21s}\\
\vdots\\
y_{is}x_{1is}x_{2is}\\
\vdots\\
y_{NS}x_{1NS}x_{2NS}
\end{pmatrix}.
\]
The model moments are
\[
m\cs{\tilde{\x}}{\th}=\frac{1}{S}\sum_{s=1}^{S}m\cs{\tilde{\x}_{s}}{\th}.
\]
Note that $S$ is usually a large number, like $10,000$.
\end{itemize}
\item $\topc$ 

\begin{itemize}
\item For SMM, we have
\[
\hat{\th}_{\mathrm{SMM}}=\arg\min_{\th}\n{m\cs{\tilde{\x}}{\th}-m(\x)}.
\]
In the $L^{2}$ norm way,
\[
\arg\min_{\th}e\cs{\tilde{\x}}{\th}\t\W e\cs{\tilde{\x}}{\th},
\]
where
\[
e\cs{\tilde{\x}}{\th}\coloneqq m\cs{\tilde{\x}}{\th}-m(\x).
\]

\end{itemize}
\item $\topc$ 

\begin{itemize}
\item Taking draws from the truncated normal distribution.

\begin{itemize}
\item Let the PDF be $\phi\cs{\x}{\th}$ and the CDF be $\Phi\cs{\x}{\th}$.
To simulate a general distribution, here are the steps:\end{itemize}
\begin{enumerate}
\item Draw $N$ values $u_{i}\sim\mathrm{Unif}\left(0,1\right)$. 
\item Use $\Phi\cs{\x}{\th}$ to convert $u_{i}$ to $x_{i}$ (the implied
values from this PDF) $\implies x_{i}\sim\phi\cs{\x}{\th}.$ \end{enumerate}
\begin{itemize}
\item \textbf{$\note$}The SMM problem will be a bonus problem.
\end{itemize}
\end{itemize}
\end{itemize}
\newpage{}

\[
\begday{Mon}{23}{Jul}{2018}
\]

\begin{itemize}
\item $\topc$ $\kw{Lucas\ Tree\ Model}$.
\item $\note$ Review of probability.

\begin{itemize}
\item Start with probability space $Z\coloneqq\left\{ z_{1},z_{2},\ldots,z_{n}\right\} $.
Assume $Z$ stays constant over time. Each $z_{i}$ is mutually exclusive
and exactly one must occur.
\item Take the infinite cartesian product of this set, $Z^{\infty}$. We
are interested in an infinite horizon. Call this set $\Omega$.
\item For $\omega\in\Omega$, we have $\omega=\left(z^{1},z^{2},z^{3},\ldots,z^{t},\ldots\right)$.
Call this a \emph{path}.
\item Random variable $X\left(\omega\right):\Omega\to\R$. 
\end{itemize}
\item $\ex$ Simple random variables.

\begin{itemize}
\item Suppose $z^{1}=z_{3}$. Then 
\begin{eqnarray*}
X(\omega) & = & \CS a{z^{1}=a}\\
 & = & z_{3}.
\end{eqnarray*}
Alternatively, let
\[
X\pr(\omega)=\CS a{z^{2}=a}.
\]
 Consider two different paths:
\[
\left(z_{1},z_{2}z_{7},z_{14},z_{2},z_{1},z_{25},\ldots\right)
\]
and
\[
\left(z_{1},z_{7},z_{14},z_{2},z_{27},\ldots\right).
\]
These paths are isomorphic to decision/probability tree paths.
\end{itemize}
\item $\defn$ A $\kw{filtration}$ is a sequence of $\sigma$-algebras
or partitions.
\item $\ex$ Consider $Z=\left\{ 1,2\right\} $. Two possible paths are
\[
\left(2,1,2,2,1,1,1,2,1,\ldots\right)
\]
and
\[
\left(1,2,2,2,1,1,2,1,\ldots\right).
\]


\begin{itemize}
\item We can partition this set into two possible sets:
\[
\CS{\omega}{z_{1}=1}
\]
and 
\[
\CS{\omega}{z_{1}=2}.
\]
If we combined these two sets with $\emptyset$ and $\Omega$, we
get a $\sigma$-algebra! Call this \textbf{S.A. 1.}
\item Now consider \textbf{S.A. 2.} What could it be? We have
\[
\CS{\omega}{z_{1}=1,z_{2}=1},\cdots,\CS{\omega}{z_{1}=2,z_{2}=2};
\]
in total we have four sets. Now include $\es$ and $\Omega$. If we
include all the unions and cross-unions, we would also get a $\sigma$-algebra.
In other words, this would be the $\sigma$-algebra \emph{generated}
by these sets. Note that this resultant $\sigma$-algebra is\textbf{
$\kw{finer}$ }than S.A. 1. In other words, S.A. 1 is $\kw{coarser}$
than S.A. 2.
\item We make a sequence of successively finer $\sigma$-algebras, as we
learn more information. This would be called a \textbf{filtration}.
\item Let $\P\left\{ 1\right\} \eqqcolon\pi(1)=\frac{1}{3}$ and $\pi(2)=\frac{2}{3}$,
and the outcomes IID. 
\end{itemize}
\item $\topc$ Consider the investor's problem,
\[
\max_{\c}\left\{ \E{\sum_{t=0}^{\infty}\beta^{t}u\left(c_{t}\right)}\right\} .
\]
Let's say we have 
\[
\begin{array}{ccc}
\mbox{Asset} & \mbox{Price} & \mbox{Dividend}\\
\hline \mbox{Asset 1} & p_{t} & d_{t}\left(\omega\right),
\end{array}
\]
where the price and the dividend are both random variables. The only
things that we can consume are with the dividends that are paid out
by these assets.

\begin{itemize}
\item Define $\th_{t}$ as the household's portfolio at time $t$. This
might be
\[
\th_{t}=\left(\theta_{1t},\theta_{2t},\ldots,\theta_{nt}\right),
\]
i.e., how many shares of each stock you have in your portfolio.
\item Households take their wealth at time $t$ and choose to eat some of
it and invest the rest of it. At time $t$, 
\[
c_{t}+p_{1t}\theta_{1t}+p_{2t}\theta_{2t}+\cdots+p_{nt}\theta_{nt}.
\]
 This represents how much the agent consumes and distributes investment
across different assets at time $t$. Note that
\[
c_{t}+p_{1t}\theta_{1t}+p_{2t}\theta_{2t}+\cdots+p_{nt}\theta_{nt}\leq\left(p_{1t}+d_{1t}\right)\theta_{1,t-1}+\cdots+\left(p_{nt}+d_{nt}\right)\theta_{n,t-1}.
\]
In vector notation,
\[
\usub{c_{t}}{\mbox{not a vector}}{}{+\p\pr_{t}\th_{t}\leq\left(\p\pr_{t}+\mathbf{d}\pr_{t}\right)\th_{t-1}.}
\]
Rewriting, we have
\begin{eqnarray*}
c_{t} & = & \p\pr_{t}\th_{t-1}+\mathbf{d}\pr_{t}\th_{t-1}+\p\pr_{t}\th_{t}\\
 & = & \p\pr_{t}\left(\th_{t-1}-\th_{t}\right)+\mathbf{d}\pr_{t}\th_{t-1}.
\end{eqnarray*}
So our problem is
\[
\usub{\p\pr_{t}\left(\th_{t-1}-\th_{t}\right)}{\begin{array}{c}
\uparrow\;\;\,\\
\mbox{\emph{change in portfolio wealth}}
\end{array}}{\max_{\th}\Bigg\{\mathrm{E}\Bigg[\sum_{t=0}^{\infty}\beta^{t}u\Big(}{+\mathbf{d}\pr_{t}\th_{t-1}\Big)\Bigg]\Bigg\}.}
\]

\item So what is this individual going to do? Take the FOCs of the above
with respect to $\th_{t}$:
\[
\usub{c_{t}}{\begin{array}{c}
\uparrow\qquad\quad\;\\
\p\pr_{t}\th_{t-1}+\mathbf{d}\pr_{t}\th_{t-1}+\p\pr_{t}\th_{t}
\end{array}}{\mathrm{E}\Bigg[\beta^{t}\Big(-u\pr\big(}{\big)\p{}_{t}+\beta u\pr\left(c_{t+1}\right)\left(\p_{t+1}+\mathbf{d}_{t+1}\right)\Big)\Bigg]=\o\t.}
\]
For row $i$:
\[
\E{\beta^{t}\left(-u\pr\left(c_{t}\right)p_{it}+\beta u\pr\left(c_{t+1}\right)\left(p_{i,t+1}+d_{i,t+1}\right)\right)}=0.
\]
Divide the LHS and the RHS by $p_{it}$: 
\[
\E{\beta^{t}\left(-u\pr\left(c_{t}\right)+\beta u\pr\left(c_{t+1}\right)\usub{\frac{p_{i,t+1}+d_{i,t+1}}{p_{it}}}{R_{it}}{}{}\right)}=0.
\]

\end{itemize}
\end{itemize}
\newpage{}

\[
\begday{Wed}{25}{Jul}{18}
\]



\section*{Stochastic Discount Factor.}
\begin{itemize}
\item $ $
\begin{eqnarray*}
\E{-\beta^{t}u\pr\left(c_{t}\right)\p_{t}+\beta^{t+1}u\pr\left(c_{t+1}\right)\left(\p_{t+1}+\dd_{t+1}\right)} & = & 0\\
\E{-u\pr\left(c_{t}\right)\p_{t}+\beta u\pr\left(c_{t+1}\right)\left(\p_{t+1}+\dd_{t+1}\right)} & = & 0\\
\implies-u\pr\left(c_{t}\right)+\beta\cE{u\pr\left(c_{t+1}\right)R_{it}}{\Omega_{t}} & = & 0\\
\implies u\pr\left(c_{t}\right) & = & \beta\cE{u\pr\left(c_{t+1}\right)R_{it}}{\Omega_{t}}.
\end{eqnarray*}
\[
1=\cE{\usub{\boxed{\beta\frac{u\pr\left(c_{t+1}\right)}{u\pr\left(c_{t}\right)}}}{\kw{stochastic\ discount\ factor}}{\mbox{}}{R_{it}}}{\Omega_{t}}.
\]
Note that for small agents, $R_{it}$ is exogenous to choices. Also,
\begin{eqnarray*}
1 & = & \cE{m_{t}R_{it}}{\Omega_{t}}\\
1 & = & \cCov{m_{t}}{R_{it}}{\Omega_{t}}+\cE{m_{t}}{\Omega_{t}}\cE{R_{it}}{\Omega_{t}}.
\end{eqnarray*}
For now, write $\cCov{m_{t}}{R_{it}}{\Omega_{t}}\eqqcolon\Cov{m_{t}}{R_{it}}$
for brevity. We have
\[
\usub{\Cov{m_{t}}{R_{it}}+\E{m_{t}}\E{R_{it}}=\Cov{m_{t}}{R_{jt}}+\E{m_{t}}\E{R_{jt}}}{\text{risk-return tradeoff}}{}{}
\]
(for $i\neq j$ in general).
\item Recall
\[
m_{t}=\beta\frac{u\pr\left(c_{t+1}\right)}{u\pr\left(c_{t}\right)}
\]
 and 
\[
R_{it}=\frac{\p_{i,t+1}+\dd_{i,t+1}}{\p_{it}}.
\]

\end{itemize}
\begin{minipage}[t]{1\columnwidth}%
\begin{shaded}%
$\note$ Small investors do not have control over $R$; they take
it as given, and $\E R$ is a belief based on information that the
investor has. Portfolio choices $\th$ are encapsulated in $m$. As
an investor, we are manipulation the $\Cov mR$ term. In equilibrium,
we all do that.\end{shaded}%
\end{minipage}
\begin{itemize}
\item consider the FOC
\[
-u\pr\left(c_{t}\right)+\beta\E{u\pr\left(c_{t+1}\right)\left(\frac{1}{\p_{it}}\right)}=0.
\]
Rearranging, we have 
\[
\p_{it}=\beta\E{\frac{u\pr\left(c_{t+1}\right)}{u\pr\left(c_{t}\right)}}.
\]
This is the $\kw{expected\ discount\ factor}$. 
\item $\kw{Gross\ return}$:
\begin{eqnarray*}
R_{it} & = & \usub{1+r_{it}}{(\ast)}{}{}\\
 & = & \frac{1}{\p_{it}}\\
 & = & \boxed{\frac{1}{\beta\E{\frac{u\pr\left(c_{t+1}\right)}{u\pr\left(c_{t}\right)}}}}.
\end{eqnarray*}
This is the implied rate of return by this model. This can be used
to estimate the parameters of the SDF through GMM. This is a \emph{$\kw{moment\ condition}$.}
Note that GMM was developed by Lars Hansen specifically to estimate
these particular moment conditions. Also note that $(\ast)$ is something
that we can easily observe. We can use all the expected marginal utilities
information to estimate what people's $\beta$ is.

\begin{itemize}
\item We would assume that
\[
u\left(c_{t+1}\right)=\frac{c_{t+1}^{1-\gamma}}{1-\gamma}.
\]
 
\end{itemize}
\item If we take the FOC
\begin{eqnarray*}
g\left(\gamma,\beta,R\right) & = & \beta\E{\frac{u\pr\left(c_{t+1}\right)}{u\pr\left(c_{t}\right)}R_{it}}-1\\
 & = & 0
\end{eqnarray*}
 as a \emph{moment condition}. We will do this on the homework. Every
period, calculate 
\[
\E{\frac{u\pr\left(c_{t+1}\right)}{u\pr\left(c_{t}\right)}R_{it}-1}.
\]
Figure out what the average is, adjust the $\gamma$, and figure out
what $\gamma$ would make this condition equal to zero. (This is Questions
1 \& 2.)
\end{itemize}

\section*{Kyle (1985) Model}
\begin{itemize}
\item Agents:

\begin{itemize}
\item $\kw{Market\ Makers}$: The ones submitting limit orders. We want
to understand this party. To these guys, $V$ is a random variable,
but they know the distribution of $V$. They don't know which traders
are informed or uninformed. They \emph{observe} $Y=X+U$ (the \emph{sum}
of informed and uninformed demand).
\item $\kw{Informed\ Traders}$: The ones who actually \emph{know} what
$V$ is exactly. They also know the distribution of $V$ that market
makers know. Their optimal demand is $X(V)$.
\item \emph{Noise Traders} or $\kw{Uninformed/Liquidity\ Traders}$: Individuals
that are trading in ways that are uncorrelated with the future value
of the asset. \emph{There is no information in this trade}. No ``market
timing'' effects involved. They know nothing. They could be bad traders,
or they could just be laypeople selling stock in order to gain idiosyncratic
personal financial liquidity. We assume $U\sim\Norm 0{\sigma_{u}^{2}}$.
\textbf{Assume that $U,V$ are uncorrelated.}
\end{itemize}
\item Let future value of asset is $V$. By ``future'' for this model,
we refer to a small time horizon, i.e., less than a week or so. Assume
$V\sim\Norm{p_{0}}{\Sigma_{0}}$. The market makers know this distribution.
\end{itemize}
\fbox{\begin{minipage}[t]{1\columnwidth}%
\uline{Price Function.}

\textbf{$\defn$ }In $\kw{equilibrium}$, a market maker sets a $\kw{price\ function}$
$P(Y)$, where $Y=X+U$. The informed traders also have a $\kw{demand}$
such that $\cE{\usub{-Y}{\text{position}}{(V-P(Y))(})}Y=0$ (competitive
risk-neutral market makers), and where $X(U)$ maximizes $\cE{(V-P(Y))X(U)}V$.
\\
\\
\emph{In general, this is a very hard system of equations to solve.}%
\end{minipage}}
\begin{itemize}
\item To solve for the equilibrium, we use \emph{guess and check}. Assume
that the equilibrium is $P(Y)=\mu+\lambda Y$. Assume linear. \emph{Note
that this is the place where you would expect to see linearity}\textemdash risk-neutral
and linear-preference assumptions. Take this as given, and move directly
to 
\[
\cE{(V-P(Y))X(U)}V.
\]
 They want to 
\[
\max_{X}\cE{\left(V-\mu-\lambda\left(X+U\right)\right)X}V.
\]
Note that for informed traders, $U$ is the only thing that is unknown.
Now some simplification:
\begin{eqnarray*}
 & = & \cE{VX-\mu X-\lambda X^{2}-\lambda UX}V\\
 & = & VX-\mu X-\lambda X^{2}-\lambda X\usub{\cE UV}{=0}{}{}\\
 & = & VX-\mu X-\lambda X^{2}.
\end{eqnarray*}
Calculating the FOC and solving for optimal $X$,
\[
V-\mu-2\lambda X=0
\]
so 
\begin{eqnarray*}
X & = & \frac{V}{2\lambda}-\frac{\mu}{2\lambda}\\
 & = & -\frac{\mu}{2\lambda}+\frac{1}{2\lambda}V.
\end{eqnarray*}
Note that the second derivative is negative if $\lambda>0$. This
is an important consideration.
\end{itemize}
\newpage{}

\[
\begday F{27}{Jul}{18}
\]



\section*{Review}
\begin{itemize}
\item NT: $U\sim\Norm 0{\sigma_{U}^{2}}$, $V\sim\Norm{p_{0}}{\Sigma_{0}},$
$Y=X+U$.
\item MM: $\cE{\left(V-P\right)\left(-Y\right)}Y$, $P(Y)=\mu+\lambda Y$ 
\item IT: 
\[
\max_{X}\cE{\left(V-P\left(X+U\right)\right)X}V
\]
\[
X=-\frac{\mu}{2\lambda}+\frac{1}{2\lambda}V.
\]
Market Makers want to know what $V$ is based on all the information
they have. They will never know $X$ or $U$ to any amount of certainty.
\item Market makers try to find
\[
\cE V{X+U=Y}.
\]
Note that 
\begin{eqnarray*}
Y & = & X+U\\
 & = & \usub{-\frac{\mu}{2\lambda}+\frac{1}{2\lambda}V}{\begin{array}{c}
\wr\\
\Norm{-\frac{\mu}{2\lambda}+\frac{1}{2\lambda}p_{0}}{\frac{1}{4\lambda^{2}}\Sigma_{0}+\sigma_{U}^{2}}
\end{array}}{}{+U\text{.}}
\end{eqnarray*}

\item What is the relationship between $Y$ and $V$?

\begin{itemize}
\item To solve, find the covariance:
\begin{eqnarray*}
\Cov YV & = & \E{\left(Y-\E Y\right)\left(V-\E V\right)}\\
 & = & \E{\left(-\frac{\mu}{2\lambda}+\frac{1}{2\lambda}V+U-\left(-\frac{\mu}{2\lambda}+\frac{1}{2\lambda}p_{0}\right)\right)\left(V-p_{0}\right)}\\
 & = & \E{\left(\frac{1}{2\lambda}\left(V-p_{0}\right)+U\right)\left(V-p_{0}\right)}\\
 & = & \E{\frac{1}{2\lambda}\left(V-p_{0}\right)^{2}+\frac{1}{2\lambda}U\left(V-p_{0}\right)}\\
 & = & \frac{1}{2\lambda}\E{\left(V-p_{0}\right)^{2}}+\frac{1}{2\lambda}\cancelto{0}{\E{U\left(V-p_{0}\right)}}\\
 & = & \frac{1}{2\lambda}\Sigma_{0}\text{.}
\end{eqnarray*}

\end{itemize}
\item The question we have in general is that, what is
\[
\cE VY\text{?}
\]
Suppose we have RVs $A$ and $B$ where $A,B\sim\Norm{\left(\E A,\E B\right)}{\left(\Cov AB,\sigma_{A}^{2},\sigma_{B}^{2}\right)}$.
Let
\[
\cE AB=\usub{\E A}{\text{initial belief}}{}{}+\usub{\frac{\Cov AB}{\sigma_{B}^{2}}\left(B-\E B\right)}{\text{update belief}}{}{}\text{.}
\]
\[
\cE VY=p_{0}+\frac{\frac{1}{2\lambda}\Sigma_{0}}{\left(\frac{1}{2\lambda}\right)^{2}\Sigma_{0}+\sigma_{U}^{2}}\left(Y-\left(-\frac{\mu}{2\lambda}+\frac{1}{2\lambda}p_{0}\right)\right)\text{.}
\]
In equilibrium, we have 
\begin{eqnarray*}
-Y\cE{V-p(Y)}Y & = & 0\\
-Y\left(\cE VY-p\left(Y\right)\right) & = & 0\\
 & \Downarrow\\
\cE VY & = & p(Y)\text{.}
\end{eqnarray*}
Note that 
\[
P(Y)=\mu+\lambda Y
\]
 and 
\[
P\left(Y\right)=\usub{p_{0}+\frac{\frac{1}{2\lambda}\Sigma_{0}}{\frac{1}{4\lambda^{2}}\Sigma_{0}+\sigma_{U}^{2}}\left(Y-\left(-\frac{\mu}{2\lambda}+\frac{1}{2\lambda}p_{0}\right)\right)}{(\dagger)}{}{}\text{.}
\]
Solving for $\lambda$ involves solving 
\[
\lambda=\frac{\frac{1}{2\lambda}\Sigma_{0}}{\frac{1}{4\lambda^{2}}\Sigma_{0}+\sigma_{U}^{2}}\text{.}
\]
Solving for $\lambda$ itself, we have
\begin{eqnarray*}
\lambda & = & \frac{\frac{1}{2\lambda}\Sigma_{0}}{\frac{1}{4\lambda^{2}}\Sigma_{0}+\sigma_{U}^{2}}\\
\left(\frac{1}{4\lambda^{2}}\Sigma_{0}+\sigma_{U}^{2}\right)\lambda & = & \frac{1}{2\lambda}\Sigma_{0}\\
\frac{1}{4\lambda}\Sigma_{0}+\lambda\sigma_{U}^{2} & = & \frac{1}{2\lambda}\Sigma_{0}\\
\lambda\sigma_{U}^{2} & = & -\frac{1}{2\lambda}\Sigma_{0}\\
\lambda & = & \sqrt{-\frac{\Sigma_{0}}{2\sigma_{U}^{2}}}\\
 & \vdots\\
 & = & \boxed{\frac{\sqrt{\Sigma_{0}}}{2\sigma_{U}}}\text{.}
\end{eqnarray*}
Note that $\mu=p_{0}$ is a solution to $(\dagger)$. Solution:
\[
P(Y)=p_{0}+\frac{\sqrt{\Sigma_{0}}}{2\sigma_{U}}Y\text{.}
\]
So,
\begin{eqnarray*}
X(U) & = & -\frac{\mu}{2\lambda}+\frac{1}{2\lambda}V\\
 & = & \frac{-p_{0}}{\left(\frac{\sqrt{\Sigma_{0}}}{\sigma_{U}}\right)}+\usub{\frac{\sigma_{U}}{\sqrt{\Sigma_{0}}}}{\text{ratio of two SDs}}{}{}V\text{.}
\end{eqnarray*}
In other words, \emph{If there's a bunch of dumb people out there,
then a smart trader doesn't have to carefully hide the aggression
of his or her trading strategies / demand of trades.}
\item Note that $p_{0}$ is the BBO and represents the intersection of demand
and supply in the capital market's limit order book.

\begin{itemize}
\item \emph{Note}: a flat LOB means that $\sigma_{U}$ is small compared
to $\Sigma_{0}$, which tells us that there is not very much noise
compared to meaningful/informed trades.
\end{itemize}
\end{itemize}
%%%%%%%%%%%%%%%%%%%%%%%%%%%%%%%%%%%%%%%%%%%%%%%%%%%%%%%%%%%%%%%%%%%%%%%%%%%%%%%%%%%%%%%%%%%%%%%%%%%%%%%%%%%%%%%%%%%%%%%%%%%%%%%%%%%%%%%%%%%%%%%%%%%%%%%%%%%%%%%%%%%%%%%%%%%%%%%%%%%%%%%%%%%%%%%%%%%%%%%%%%%%%%%%%%%%%%%%%%%%%%%%%%%%%%%%%%%%%%%%%%%%%%%%%%%%%%%%%%%%%%%%%%%%%%%%%%%%%%%%%%%%%%%%%%%%%%%%%%%%%%%%%%%%%%%%%%%%%%%%%%%%%%%%%%%%%%%%%%%%%%%%%%%%%%%%%%%%%%%%%%%%%%%%%%%%%%%%%%%%%%%%%%%%%%%%%%%%%%%%%%%%%%%%%%%%%%%%%%%%%%%%%%%%%%%%%%%%%%%%%%%%%%%%%%%%%%%%%%%%%%%%%%%%%%%%%%%%%%%%%%%%%%%%%%%%%%%%%%%%%%%%%%%%%%%%%%%%%%%%%%%%%%%%%%%%%%%%%%%%%%%%%%%%%%%%%%%%%%%%%%%%%%%%%%%%%%%%%%%%%%%%%%%%%%%%%%%%%%%%%%%%%%%%%%%%%%%%%%%%%%%%%%%%%%%%%%%%%%%%%%%%%%%%%%%%%%%%%%%%%%%%%%%%%%%%%%%%%%%%%%%%%%%%%%%%%%%%%%%%%%%%%%%%%%%%%%%%%%%%%%%%%%%%%%%%%%%%%%%%%%%%%%%%%%%%%%%%%%%%%%%%%%%%%%%%%%%%%%%%%%%%%%%%%%%%%%%%%%%%%%%%%%%%%%%%%%%%%%%%%%%%%%%%%%%%%%%%%%%%%%%%%%%%%%%%%%%%%%%%%%%%%%%%%%%%%%%%%%%%%%%%%%%%%%%
\end{document}
