%% LyX 2.1.4 created this file.  For more info, see http://www.lyx.org/.
%% Do not edit unless you really know what you are doing.
\documentclass[english]{article}
\usepackage[latin9]{inputenc}
\usepackage{geometry}
\geometry{verbose,tmargin=1in,bmargin=1in,lmargin=1in,rmargin=1in}
\usepackage{color}
\definecolor{shadecolor}{rgb}{1, 0.9375, 0.929688}
\usepackage{framed}
\usepackage{amsmath}

\makeatletter

%%%%%%%%%%%%%%%%%%%%%%%%%%%%%% LyX specific LaTeX commands.
%% Because html converters don't know tabularnewline
\providecommand{\tabularnewline}{\\}

%%%%%%%%%%%%%%%%%%%%%%%%%%%%%% User specified LaTeX commands.
\usepackage{fancyhdr}
\pagestyle{fancy}
%\usepackage[proportional,scaled=1.064]{erewhon}
%\usepackage[erewhon,vvarbb,bigdelims]{newtxmath}
%\usepackage[T1]{fontenc}
%\renewcommand*\oldstylenums[1]{\textosf{#1}}
\lhead{\bf{Harrison Beard}}
\rhead{OSM Boot Camp: \bf{Math ProbSet1}}
\cfoot{\thepage}

\makeatother

\usepackage{babel}
\begin{document}
\global\long\def\norm#1{\left\Vert #1\right\Vert }
\global\long\def\eval#1{\left.#1\right|}
\global\long\def\R{\mathbb{R}}
\global\long\def\N{\mathbb{N}}
\global\long\def\Q{\mathbb{Q}}
\global\long\def\c{^{\complement}}
\global\long\def\pow#1{\mathcal{P}\left(#1\right)}
\global\long\def\es{\mbox{\ensuremath{\emptyset}}}
\global\long\def\pr{^{\prime}}
\global\long\def\part#1#2{\frac{\partial#1}{\partial#2}}
\global\long\def\sm{\smallsetminus}
\global\long\def\usub#1#2#3#4{\underset{#2}{#3\underbrace{#1}#4}}
\global\long\def\E{\mathrm{E}}
\global\long\def\Var#1{\mathrm{Var}\left(#1\right)}
\global\long\def\e#1{\mathrm{e}^{#1}}
\global\long\def\G#1{\Gamma\left(#1\right)}
\global\long\def\P{\mathrm{P} }
\global\long\def\CS#1#2{\left\{  \left.#1\phantom{\mathllap{#2}}\right|#2\right\}  }
\global\long\def\i#1#2#3#4{\int_{#2}^{#3}#1\,\mathrm{d}#4}
\global\long\def\diff#1#2{\frac{\mathrm{d}#1}{\mathrm{d}#2}}
\global\long\def\F#1#2{\mathrm{F}\left(#1,#2\right)}
\global\long\def\iid{\stackrel{\mbox{iid}}{\sim}}
\global\long\def\Norm#1#2{\mathcal{N}\left(#1,#2\right)}
\global\long\def\t#1{\mathrm{t}\left(#1\right)}
\global\long\def\hyp#1#2#3{\left\{  \begin{array}{rl}
 H_{0}:  &  #1\\
 H_{1}:  &  #2\\
 \alpha=  &  #3 
\end{array}\right.}
\global\long\def\hyptests#1#2{\left\{  \begin{array}{rl}
 H_{0}:  &  #1\\
 H_{1}:  &  #2 
\end{array}\right.}
\global\long\def\v#1{\vec{\mathbf{#1}}}
\global\long\def\Question{\textbf{\textsf{Question.}}\quad}
\global\long\def\Answer{\textbf{\textsf{Answer.}}\quad}



\title{\textsf{OSM Boot Camp: }\textsf{\textbf{Econ ProbSet1}}}


\author{\textsf{Harrison Beard}}


\date{\textsf{25 June 2018}}

\maketitle

\section*{\textsf{\textcolor{red}{Exercise 1.}}}

\begin{minipage}[t]{1\columnwidth}%
\begin{shaded}%
$\Question$Consider the problem of the owner of an oil field. The
owner has $B$ barrels of oil. She can sell these barrels at price
$p$, at time $t$. Her objective is to maximize the discounted present
value of sales of oil\textemdash we'll assume there are no extraction
costs. The owner discounts the future at a rate given by $\frac{1}{1+r}$
(where $r$ is the real interest rate and assumed to be constant).
Answer the following:
\begin{enumerate}
\item What are the state variables?
\item What are the control variables?
\item What does the transition equation look like?
\item Write down the sequence problem of the owner. Write down the Bellman
equation.
\item What does the owner's Euler equation look like?
\item What would the solution of the problem look like if $p_{t+1}=p_{t}$
for all $t$? What would the solution look like if $p_{t+1}>(1+r)p_{t}$
for all $t$? What is the condition on the path of prices necessary
for an interior solution (where the owner will extract some, but not
all, of the oil)?
\end{enumerate}
\emph{Tips:}
\begin{enumerate}
\item \emph{No need to use a computer here\textemdash This equation wants
you to apply your theory of dynamic programming.}
\item \emph{Pay attention to binding constraints.}\end{enumerate}
\end{shaded}%
\end{minipage}\\
\\
\\
$\Answer$
\begin{enumerate}
\item The state variable is $B_{t}$ (how many barrels of oil left at time
$t$) and the parameter is $r$ (the constant interest rate).
\item The control variables are the price $p_{t}$ (and $t$, the time at
which to sell the barrel). And, say that $b_{t}$ is the amount of
barrels sold in period $t$.
\item Transition equation: $B\pr=B-b$. The amount of barrels of oil left
tomorrow is the amount left today, minus the amount of barrels sold.
\item The sequence problem: 
\[
V_{T}(B)=\max_{\left\{ \left(p_{1},\ldots,p_{T}\right),\left(b_{1},\ldots,b_{T}\right)\right\} }\sum_{t=1}^{T}\left(\frac{1}{1+r}\right)^{t}\pi(b_{t},p_{t}),
\]
where $\pi(b_{t},p_{t})$ is our profit function, which we can represent
as $p_{t}b_{t}$. So we have
\[
V_{T}(B)=\max_{\left\{ \left(p_{1},\ldots,p_{T}\right),\left(b_{1},\ldots,b_{T}\right)\right\} }\sum_{t=1}^{T}\frac{p_{t}b_{t}}{\left(1+r\right)^{t}},
\]
such that $B=\sum_{t=1}^{T}b_{t}$. The Bellman Equation would be
\[
V_{T+1}(B)=\max_{\left\{ p_{1},b_{1}\right\} }\left\{ p_{1}b_{1}+V_{T}(B-b_{1})\right\} .
\]

\item Setting up the Lagrangian: 
\[
{\cal L}=\max_{\left\{ \left(p_{1},\ldots,p_{T}\right),\left(b_{1},\ldots,b_{T}\right)\right\} }\sum_{t=1}^{T}\frac{p_{t}b_{t}}{\left(1+r\right)^{t}}-\lambda_{1}\left(B-\sum_{t=1}^{T}b_{t}\right)-\lambda_{2}\left(p_{t}\right)
\]
Taking some FOCs:
\begin{eqnarray*}
\part{{\cal L}}{b_{1}} & = & \frac{p_{1}}{\left(1+r\right)}+\lambda_{1}\implies-\lambda_{1}=\frac{p_{1}}{\left(1+r\right)}\\
\part{{\cal L}}{b_{2}} & = & \frac{p_{2}}{\left(1+r\right)^{2}}+\lambda_{1}\implies-\lambda_{1}=\frac{p_{2}}{\left(1+r\right)^{2}}\\
 & \vdots
\end{eqnarray*}
From here, we can find the Euler equation:
\[
\frac{p_{1}}{\left(1+r\right)}=\frac{p_{2}}{\left(1+r\right)^{2}}\implies p_{2}=(1+r)p_{1}.
\]
Note that this relationship holds for all $p_{t}$ and $p_{t+1}$
for $t\in\left\{ 1,\ldots,T-1\right\} .$ So in general, our Euler
equation is
\[
p_{t+1}=(1+r)p_{t}.
\]

\item What would the solution of the problem look like if $p_{t+1}=p_{t}$
for all $t$? What would the solution look like if $p_{t+1}>(1+r)p_{t}$
for all $t$? What is the condition on the path of prices necessary
for an interior solution (where the owner will extract some, but not
all, of the oil)?\end{enumerate}
\begin{itemize}
\item If $p_{t+1}=p_{t}$ for all $t$, the solution would be sell a larger
amount at the beginning than later on (in fact, sell all of them in
the first period, since there's no intertemporal utility constraint
to even out the selling of oil), since compounded interest would erode
away the present-value price of barrels for large $t$. 
\item If $p_{t+1}>(1+r)p_{t}$, then the solution would be to sell more
barrels in the future (in fact, sell all of them in the last period,
for similar reasons as mentioned above), since the price mark-up effect
would exceed the compounded interest effect when discounting back
to present-value prices.
\item The owner would extract some, but not all, of the oil (achieve an\emph{
interior solution}) if the Inada condition on the path of prices was
put in place.
\end{itemize}
\[
\]



\section*{\textsf{\textcolor{red}{Exercise 2.}}}

\begin{minipage}[t]{1\columnwidth}%
\begin{shaded}%
$\Question$The Neoclassical Growth Model is a workhorse model in
macroeconomics. The problem for the social planner is to maximize
the discounted expected utility for agents in the economy:
\[
\max_{\left\{ c\right\} _{t=0}^{\infty}}\mathrm{E}_{0}\sum_{t=0}^{\infty}\beta^{t}u(c_{t}).
\]


The resource constraint is given as: 
\[
y_{t}=c_{t}+i_{t}.
\]
The law of motion for the capital stock is:
\[
k_{t+1}=(1-\delta)k_{t}+i_{t}.
\]
Output is determined by the aggregate production function:
\[
y_{t}=z_{t}k_{t}^{\alpha}.
\]
Assume that $z_{t}$ is stochastic. In particular, it is an i.i.d.
process distributed as $\log(z)\sim\Norm 0{\sigma_{z}}$.
\begin{enumerate}
\item What is (are) the state variables(s)?
\item What are the control variables?
\item Write down the Bellman Equation that represents this sequence problem.
\item Solve the growth model given the following parameterization (you may
use VFI or PFI):
\end{enumerate}
\begin{center}
\begin{tabular}{|c|c|c|}
\hline 
Parameter & Description & Value\tabularnewline
\hline 
\hline 
$u(c)=\frac{c^{1-\gamma}}{1-\gamma}$ & CRRA utility & \tabularnewline
\hline 
$\gamma$ & Coefficient of Relative Risk Aversion & 0.5\tabularnewline
\hline 
$\beta$ & Discount factor & 0.96\tabularnewline
\hline 
$\delta$ & Rate of physical depreciation & 0.05\tabularnewline
\hline 
$\alpha$ & Curvature of production function & 0.4\tabularnewline
\hline 
$\sigma_{z}$ & SD of productivity shocks & 0.2\tabularnewline
\hline 
\end{tabular}
\par\end{center}
\begin{itemize}
\item Plot the value function.
\item Plot the policy function for the choice of consumption.
\item Plot the policy function for the choice of capital next period.
\end{itemize}
\emph{Tips:}
\begin{enumerate}
\item \emph{The fact that the shocks are i.i.d. makes the computation simpler.
Consider integrating over a Monte Carlo simulation of the shocks to
find the expected values needed.}
\item \emph{Write functions for the utility function and the production
function\textemdash and depending on your solution method, variants
on these as well such as the marginal utility function.}
\item \emph{Be careful when it's possible that infeasible values of $c_{t}$
of $k_{t}$ may be chosen in your solution method.}\end{enumerate}
\end{shaded}%
\end{minipage}\\
\\
\\
$\Answer$
\begin{enumerate}
\item The state variable is $k_{t}$, the capital stock; the parameters
are $\beta,\delta,\alpha,\sigma_{z},k_{t}$: the discount rate, the
depreciation rate, the capital intensity, and the SD of productivity
shocks.
\item The control variables are $c_{t}$ and $i_{t}$: the consumption and
investment.
\item The Bellman equation can be represented as 
\[
V_{T+1}(k_{0})=\max_{c_{0},i_{0}}\left\{ u(c_{0})+V_{T}(c_{1})\right\} ,
\]
where $c_{1}=y_{1}-i_{1}$ and $y_{1}=z_{1}k_{1}^{\alpha}$, and $k_{1}=(1-\delta)k_{0}+i_{0}$,
so we have that
\begin{eqnarray*}
c & = & y-i\\
 & = & zk^{\alpha}-i\\
 & = & zk^{\alpha}-(k\pr-(1-\delta)k).
\end{eqnarray*}
\[
V_{T+1}(k_{0})=\max_{c_{0},i_{0}}\left\{ u(c_{0})+\mathrm{E}_{0}\sum_{t=0}^{\infty}\beta^{t}u\left(z_{1}\left((a-\delta)k_{0}+i_{0}\right)^{\alpha}-i_{1}\right)\right\} ;
\]
note that $V_{T}(c_{1})=\E_{0}\sum_{t=0}^{\infty}\beta^{t}u(c_{t})$.
\item Solve the growth model given the following parameterization (you may
use VFI or PFI):
\end{enumerate}
\begin{center}
\begin{tabular}{|c|c|c|}
\hline 
Parameter & Description & Value\tabularnewline
\hline 
\hline 
$u(c)=\frac{c^{1-\gamma}}{1-\gamma}$ & CRRA utility & \tabularnewline
\hline 
$\gamma$ & Coefficient of Relative Risk Aversion & 0.5\tabularnewline
\hline 
$\beta$ & Discount factor & 0.96\tabularnewline
\hline 
$\delta$ & Rate of physical depreciation & 0.05\tabularnewline
\hline 
$\alpha$ & Curvature of production function & 0.4\tabularnewline
\hline 
$\sigma_{z}$ & SD of productivity shocks & 0.2\tabularnewline
\hline 
\end{tabular}
\par\end{center}
\begin{itemize}
\item Plot the value function.
\item Plot the policy function for the choice of consumption.
\item Plot the policy function for the choice of capital next period.
\end{itemize}
\[
\]



\section*{\textsf{\textcolor{red}{Exercise 3.}}}

\begin{minipage}[t]{1\columnwidth}%
\begin{shaded}%
$\Question$Use the same neoclassical growth model as above, but consider
the case where there is serial correlation in the productivity shock.
In particular, assume that $z_{t}$ is given by:
\[
\log(z_{t})=\rho\log(z_{t-1})+v_{t}
\]
where $v_{t}\sim\Norm 0{\sigma_{v}}$. Let $\rho=0.8$ and $v=0.1$.
\begin{enumerate}
\item Write down the Bellman Equation that represents the planner's problem
in this case.
\item Approximate the $\mathrm{AR}(1)$ process with a Markov chain and
solve the model:\end{enumerate}
\begin{itemize}
\item Plot the value function for at least 3 values of the productivity
shock.
\item Plot the policy function for the choice of consumption for at least
3 values of the productivity shock.
\item Plot the policy function for the choice of capital next period for
at least 3 values of the productivity shock.
\end{itemize}
\emph{Tips:}
\begin{enumerate}
\item \emph{Use $\mathtt{quantecon.markov.approximate}$ or the $\mathtt{ar1\_approximate.py}$
module to approximate the $\mathrm{AR}(1)$ process.}\end{enumerate}
\end{shaded}%
\end{minipage}\\
\\
\\
$\Answer$

\[
\]



\section*{\textsf{\textcolor{red}{Exercise 4.}}}

\begin{minipage}[t]{1\columnwidth}%
\begin{shaded}%
$\Question$The search and matching model of labor markets is a key
model in the macroeconomic labor literature. In one version of this
model, potential workers receive wage offers from a distribution of
wages in each period. Potential workers must decide whether to accept
and begin work at this age (and work at this age forever) or decline
the offer and continue to ``search'' (i.e., receive wage offers
from some exogenous distribution).

The potential workers seek to maximize the expected, discounted sum
of earnings:
\[
\E_{0}\sum_{t=0}^{\infty}\beta^{t}y_{t}.
\]
Income, $y_{t}$, is equal to $w_{t}$ if employed. If unemployed,
agents receive unemployment benefits $b$.

Assume that wage offers are distributed as $\log(w_{t})\sim\Norm{\mu}{\sigma}$.
\begin{enumerate}
\item Write down the Bellman equation representing this optimal stopping
problem.
\item Solve the model, using the following parameterization:
\end{enumerate}
\begin{center}
\begin{tabular}{|c|c|c|}
\hline 
Parameter & Description & Value\tabularnewline
\hline 
\hline 
$\beta$ & Rate of time preference & 0.96\tabularnewline
\hline 
$b$ & Unemployment benefits & 0.05\tabularnewline
\hline 
$\mu$ & Mean of log wages & 0.0\tabularnewline
\hline 
$\sigma$ & SD of wage draws & 0.15\tabularnewline
\hline 
\end{tabular}
\par\end{center}
\begin{itemize}
\item Plot the value function.
\item Find the ``reservation wage'' for the unemployed worker (i.e., the
wage that makes her indifferent between accepting the job offer and
not)
\item Vary $b$ from $0.5$ to $1.0$ and plot the reservation wage for
each value of $b$. How do unemployment benefits affect the reservation
wage?\end{itemize}
\end{shaded}%
\end{minipage}\\
\\
\\
$\Answer$
\end{document}
